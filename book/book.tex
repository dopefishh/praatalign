%&book
\begin{document}
\cleardoublepage
\maketitle
\setcounter{page}{1}
\tableofcontents
\chapter{Introduction}
\section{Introduction}
Praatalign is a plug-in for Praat that can be used to do forced phonetic
alignment on speech signals and in particular free speech. Praatalign combines
the powerful HTK toolkit with the interactivity and modularity of Praat to
create an interactive, intuitive and extendable application. Text in
\texttt{monospace} means that the word is a command, variable or value.

\section{Example workflow}
\begin{figure}[H]
	\centering
	\includegraphics[width=0.5\linewidth]{s1.eps}
	\caption{Opening and selecting a \texttt{LongSound} and a \texttt{TextGrid}}
\end{figure}

\begin{figure}[H]
	\centering
	\includegraphics[width=\linewidth]{s2.eps}
	\caption{Press \texttt{View \& Edit}}
\end{figure}

\begin{figure}[H]
	\centering
	\includegraphics[width=\linewidth]{s2a.eps}
	\caption{Press \texttt{Set up forced alignment\ldots}}
\end{figure}

\begin{figure}[H]
	\centering
	\includegraphics[scale=0.6]{s3.eps}
	\caption{Edit the settings to your liking}
\end{figure}

\begin{figure}[H]
	\centering
	\includegraphics[scale=0.6]{s4.eps}
	\caption{Because of the dictionary flag a dialog spawns}
\end{figure}

\begin{figure}[H]
	\centering
	\includegraphics[width=\linewidth]{s5.eps}
	\caption{Select the dictionary}
\end{figure}

\begin{figure}[H]
	\centering
	\includegraphics[width=\linewidth]{s6.eps}
	\caption{Select an interval, zoom in and press
	\texttt{Align the current interval}}
\end{figure}

\begin{figure}[H]
	\centering
	\includegraphics[width=\linewidth]{s7.eps}
	\caption{Wait a couple of seconds and see the alignment. Note that there is
an error because of a reduction from \texttt{cuidado} to \texttt{cuida}.}
\end{figure}

\begin{figure}[H]
	\centering
	\includegraphics[width=\linewidth]{s8.eps}
	\caption{We fix the annotation error with the reduction to
\texttt{cuida(do)}, using tructation specified in the Spanish phonetizer. Then
we run the alignment again and the results are much better. If this is a common
truncation we could add a pronunciation variant or if it is really general we
could create a ruleset file}
\end{figure}

\chapter{Installation}
\section{Preparation}
The installation of the program is very straightforward, however installing the
dependencies might not be on some systems.
Some programs are not included in the package due to licencing and environment
compatibility but they do need to be installed in order for the plug-in to work
properly. All programs Praatalign depends on are free and open source. The
following list of programs need to be installed.
\begin{itemize}
	\item \textbf{Praat}\\
		Praat is a program that allows you to do phonetic analysis and annotations
		with a computer and praatalign uses Praat to provide an interactive user
		interface to the annotated sound files. Praat can be downloaded
		here\footnote{\url{http://www.fon.hum.uva.nl/praat}}.
	\item \textbf{Python}\\
		Python is used to interpret the scripts that run the core of the aligner.
		Python, in this case Python 2, can be downloaded
		here\footnote{\url{https://www.python.org/download/}}. Be sure to download
		the Python 2 version as the Python 3 version is not supported. The scripts
		are tested on Python $2.7.x$ older versions may not work correctly.
	\item \textbf{SoX}\\
		For processing the sound files in a very detailed and controlled way we
		use SoX. Although Praat also has sound processing capabilities SoX works
		better is some situations, this is because Praat does not allow you to
		specify certain options like sampling rate for all formats. Sox can be
		downloaded here\footnote{\url{http://sox.sourceforge.net/}}.
	\item \textbf{Hcopy \& HVite}\\
		HCopy and HVite are programs from the HTK toolkit and due to licencing
		issues we can not provide the binaries in a direct way. The program is for
		free but you are not allowed to distribute it. For Windows there is a
		compiled binary package, for Linux and Mac you have to compile the package
		yourself. For windows we tested the version that is available after
		registering here\footnote{\url{%
http://htk.eng.cam.ac.uk/ftp/software/htk-3.3-windows-binary.zip}}.
		For Linux and Mac the latest compiled sources that work can be found
		here\footnote{\url{http://htk.eng.cam.ac.uk/ftp/software/HTK-3.4.tar.gz}}.
		Installing HTK is probably the hardest part of using the program, if it
		does not work you can always contact us.
\end{itemize}

\section{Installation}
\label{sec:installation}
The installation of the plug-in is very easy but the method differs for
\textit{*NIX} systems\footnote{The plugin is tested on Linux and Mac. Solaris,
BSD or any other \textit{*NIX} system should also work but are untested.}
and \textit{Windows} systems\footnote{The plugin is tested on Windows 7 and
Windows Server 2008 via Windows Terminal Services. Other versions of windows
should also work but are untested.}.
\subsection{Automated installation}
Run the installation script for your system.
\begin{itemize}
	\item \textbf{\textit{*NIX} systems}: \texttt{install.sh}\\
		Depending on the operating system you either have to run the script from
		the terminal or double click it from some explorer like program. Running
		the script from the terminal is very easy. Just start a terminal program
		and type the full and exact path in the terminal and press enter. For
		example on a Mac this will something like be: \texttt{%
		/Users/frobnicator/Downloads/praatalign/install.sh}.
	\item \textbf{\textit{Windows} systems}: \texttt{install\_win.bat}\\
		This installation script can be double clicked from the explorer and it
		will install Praatalign.
\end{itemize}

\subsection{Manual installation}
Copy the contents of the root directory to(you have to create the directory if
it does not already exist):
\begin{itemize}
	\item \textbf{\textit{*NIX} systems except \textit{Mac} systems}\\
		\texttt{\$\{HOME\}/.praat-dir/plugin\_pralign/}.
	\item \textbf{\textit{Windows} systems}\\
		\texttt{\%USERPROFILE\%\textbackslash Praat\textbackslash plugin\_pralign/}
	\item \textbf{\textit{Mac} systems}\\
		\texttt{\$\{HOME\}/Library/Preferences/Praat Prefs/plugin\_pralign/}
\end{itemize}

\chapter{Documentation}
\section{General information}
With the Praatalign plugin you can currently align out of the box the data
using Spanish, Tzeltal, English or Dutch. Presets for
Australian English, Estonian, German, Hungarian, Italian, New Zealand English
Polish and Portuguese at minimal will be added in the future. When you want a
language from the above list implemented with priority you can always contact
us. If you want other languages you can still contact us and it might be
possible using the general SAMPA model.

Dictionary, ruleset and all other files are, and should be, encoded in
\texttt{UTF-8}. To enforce this the plug-in changes the default behaviour of
Praat every time Praat loads to make sure Praats reading and writing
preferences are set to \texttt{UTF-8}.

When the plug-in is successfully installed several menu items are added in the
\texttt{TextGrid} editor under the \texttt{Interval} menu. The added
functionality only works when you are editing a \texttt{TextGrid} and a
\texttt{LongSound}. It does not work with a normal \texttt{Sound}, this is
because \texttt{Sound} objects are loaded in memory and thus detached from the
real sound file on disk and our plugin needs the sound file on disk. Currently
the plugin is only tested on \texttt{WAVE} files. It should however work on all
sound filetypes SoX can detect from the extension.

\section{Menu items}
\begin{itemize}
	\item \texttt{Generate dictionary from tier}\\
		This functions allows the user to generate a dictionary containing all the
		missing or unphonetizable words from the currently selected tier using the
		current settings the plugin is initialized with. The plugin will prompt you
		after pressing the button for a location for the dictionary file. When this
		process is done the user can add the pronunciations after every entry that
		is found in the skeleton dictionary. Note that there is no sanitation
		applied on the words. This means that if the phonetizer removes punctuation
		it can still be present in the dictionary.

		For this function to work \texttt{Set up forced alignment\ldots} has to run
		at least once.
	\item \texttt{Clean selection}\\
		This function is a helper function to clean up old or wrong alignment. When
		the function runs all annotation data within the selection within the
		selected tier will be removed. Note that this is not necessary to do before
		an alignment because this function runs by default before the alignment of
		an interval.
	\item \texttt{Align current interval}\\
		This function aligns the current selected interval on the current selected
		tier. When selecting a small interval it should not take much time at all.
		Be careful that you do not try to align an output tier from a previous
		alignment(word or phone level). When this happens you will be prompted to
		make sure you mean to align an output tier.

		For this function to work \texttt{Set up forced alignment\ldots} has to run
		at least once.
	\item \texttt{Align current tier}\\
		This function aligns the entire selected tier. Note that this can take
		quite a long time for long data. When you do not have a lot of
		pronunciation variants it still can take about $30$ minutes for an hour
		long conversation. This function will clear the tier entirely before
		aligning.

		For this function to work \texttt{Set up forced alignment\ldots} has to run
		at least once.
	\item \texttt{Set up forced alignment\ldots}\\
		When you run this function two option menus are spawned that will generate
		the necessary settings files. For almost all functions this functions has
		to run at least once before they can run. When you close the form a
		settings file is written to disk.

		\textbf{Basic options}	
		\begin{itemize}
			\item \texttt{new}\\
				Name of the tier where the phone level alignment is stored, this can be
				either an existing tier or a non existing tier. If the tier does not
				exist it will be created upon doing the first alignment. If the tier
				does exist the annotations that overlap with the alignment will be
				removed.
			\item \texttt{wrd}\\
				Name of the tier that stores the word level alignment, this can be
				again either an existing tier or a non existing tier with the same
				consequences as for the \texttt{new} value. In theory this tier can be
				the same tier as the \texttt{new} tier, however this can and will
				result in unexpected behaviour and will generate a warning every time
				you align something.
			\item \texttt{lan}\\
				Language used for the forced alignment, all properly added languages
				will appear in this drop down list. Currently only Spanish, Tzeltal,
				Dutch and English are supported. There is also a language that can be
				used for any language mapped onto the available SAMPA phones.
			\item \texttt{dic}\\
				Flag for setting a dictionary file, when you tick this box you
				will be prompted to locate the dictionary file. When a dictionary was
				previously set an extra variable is shown named \texttt{dictionary}
				containing the path to the current dictionary. When you want to change
				the dictionary you can either tick the \texttt{dic} box or change the
				path in the \texttt{dictionary} field. To unset a dictionary you can
				just clear out the \texttt{dictionary} field and leave the \texttt{dic}
				box unticked.
			\item \texttt{rul}\\
				Flag for setting a ruleset file, when you tick this box you will be
				prompted to locate the ruleset file. When a ruleset file was
				previously set an extra variable is shown named \texttt{ruleset}
				containing the path to the current ruleset file. When you want to
				change the ruleset file you can either tick the \texttt{rul} box or
				change the path in the \texttt{ruleset} field. To unset a ruleset file
				you can just clear out the \texttt{ruleset} field and leave the
				\texttt{rul} box unticked.
			\item \texttt{thr}\\
				Extra margin used for every annotation, when the annotations are placed
				to close to the real sound the initial pause can clobber up the
				beginning of speech and that can reduce the aligners performance.
				Setting the \texttt{thr} value to $0.1$ will for example increase all
				boundaries from annotations with $100$ms. Note that this does not
				change the original annotation and it will only increase the widen the
				annotation when there is room to do so, meaning that it will not create
				overlap with other annotations.
		\end{itemize}

		\textbf{Advanced options}
		\begin{itemize}
			\item \texttt{log}\\
				Location of the log output, when an annotation is aligned a log is
				produced that contains detailed output of all the subcommands and can
				be used for debugging purposes. You can either put a file path in here
				that will become the log file or if you want to discard the log you can
				put \texttt{/dev/null} there on Linux or Mac and \texttt{nul} on
				Windows.
			\item \texttt{sox}\\
				Flag for setting a custom SoX executable location, when you tick this
				box you will be prompted to locate a SoX executable. When a SoX
				executable is already present in one of the locations in
				\texttt{\$PATH} on Linux or mac or \texttt{\%PATH\%} on Windows you do
				not have to set a custom location. When a custom location has been
				previously set an extra variable named \texttt{soxex} is shown that
				contains the current custom SoX location. When you want to change this
				you can either tick the \texttt{sox} box or change the path in the
				\texttt{soxex} field.  When you want to unset a custom SoX location you
				can just clear out the \texttt{soxex} field and leave the \texttt{sox}
				box unticked.
			\item \texttt{hvite}\\
				Flag for setting a custom HVite executable location, when you tick this
				box you will be prompted to locate a HVite executable. When a HVite
				executable is already present in one of the locations in
				\texttt{\$PATH} on Linux or mac or \texttt{\%PATH\%} on Windows you do
				not have to set a custom location. When a custom location has been
				previously set an extra variable named \texttt{hviteex} is shown that
				contains the current custom HVite location. When you want to change
				this you can either tick the \texttt{hvite} box or change the path in
				the \texttt{hviteex} field. When you want to unset a custom HVite
				location you can just clear out the \texttt{hviteex} field and leave
				the \texttt{hvite} box unticked.
			\item \texttt{hcopy}\\
				Flag for setting a custom HCopy executable location, when you tick this
				box you will be prompted to locate a HCopy executable. When a HCopy
				executable is already present in one of the locations in
				\texttt{\$PATH} on Linux or mac or \texttt{\%PATH\%} on Windows you do
				not have to set a custom location. When a custom location has been
				previously set an extra variable named \texttt{hcopyex} is shown that
				contains the current custom HCopy location. When you want to change
				this you can either tick the \texttt{hcopy} box or change the path in
				the \texttt{hcopyex} field. When you want to unset a custom HCopy
				location you can just clear out the \texttt{hcopyex} field and leave
				the \texttt{hcopy} box unticked.
			\item \texttt{python}\\
				Flag for setting a custom python executable location, when you tick
				this box you will be prompted to locate a python executable. When a
				python executable is already present in one of the locations in
				\texttt{\$PATH} on Linux or mac or \texttt{\%PATH\%} on Windows you do
				not have to set a custom location. When a custom location has been
				previously set an extra variable named \texttt{pythonex} is shown that
				contains the current custom python location. When you want to change
				this you can either tick the \texttt{python} box or change the path in
				the \texttt{pythonex} field. When you want to unset a custom python
				location you can just clear out the \texttt{pythonex} field and leave
				the \texttt{python} box unticked.
		\end{itemize}
\end{itemize}

\section{Dictionary}
To phonetize words praatalign either uses the provided phonetizer or a
dictionary. Dictionaries are plain text files that contain words and one or
more pronunciations. A dictionary file is a \texttt{UTF-8} encoded file
containing non-empty lines separated by a newline
character\footnote{\label{fn:n1}On Linux and Mac this is default, on Windows
this can cause problems. When using praatalign on windows please refrain to a
text editor that has newline capabilities like Notepad++}. Lines starting with
a \texttt{\#} will be ignored and can thus be used as comments. The format of a
dictionary entry is a word followed by a tab followed by tab separated
pronunciations. An example dictionary can be found in
Listing~\ref{lst:exampledictionary}

\begin{lstlisting}[caption={Example dictionary},label={lst:exampledictionary}]
# This is comment
# This is a word with two possible pronunciations
ado<TAB>a d o<TAB>a o
# These are words with one possible pronunciation
empatar<TAB>e m p a t a r
empataran<TAB>e m p a t a r a n
\end{lstlisting}
	
\section{Ruleset}
Besides generating pronunciation by using the dictionary and phonetization you
can also use rulesets to define pronunciation variants. Ruleset make you able
to define general rules applied over all words(phonetized words and dictionary
words). In this way you can easily define for example deletion rules.
A Ruleset file is a \texttt{UTF-8} encoded
file containing non-empty lines separated by a newline character. Lines
starting with a \texttt{\#} will be ignored and can thus be used as comments.
There are two ways of defining rules for a ruleset.
\begin{itemize}
	\item \textbf{Simple}\\
		Simple rules are just find and replace queries. The first column is the
		target and the second column is the replace value. For example the deletion
		rule \texttt{a d o -> a o} can be written as \texttt{a d o<TAB>a o}.
	\item \textbf{Regular}\\
		Regular rules are like regular expressions and are therefore much more
		expressive but also more complicated to write. Regular rules start with a
		single tab character to denote that they are in fact regular expressions.
		For example a deletion rule that deletes a \texttt{d} between two vowels
		and potentially also across word boundaries can be written like \texttt{%
		<TAB>([aouie] \#?) d ([aouie])<TAB>\textbackslash 1 \textbackslash 2}.
		Internally we use the \texttt{re.sub} function from the Python \texttt{re}
		library\footnote{\url{https://docs.python.org/2/library/re.html}}. Besides
		that there are some extra options:
		\begin{itemize}
			\item \texttt{\textbackslash v} for vowels(\texttt{[aoeiu]})
			\item \texttt{\textbackslash c} for consonants(%
				\texttt{[\textasciicircum aoeiu]})
			\item \texttt{\#} for a inter word silence
		\end{itemize}
\end{itemize}

\section{Supported languages information}
Different languages have different possibilities within the possibilities of
phonetizing and dictionaries. Some languages are impossible to phonetize and
thus need a complete dictionaries. Some languages can be fully phonetized and
only need a dictionary for foreign words.

\subsection{Dutch}
A detailed description of the Dutch phones can be found in
Section~\ref{sec:sldutch}. Further specification:
\begin{table}[H]
	\caption{Dutch language properties}
	\begin{tabular}{ll}
		Phonetizer support & None\\
		Ruleset support & Full\\
		Dictionary support & Full
	\end{tabular}
\end{table}

\subsection{English}
A detailed description of the English phones can be found in
Section~\ref{sec:slenglish}
\begin{table}[H]
	\caption{Dutch language properties}
	\begin{tabular}{ll}
		Phonetizer support & None\\
		Ruleset support & Full\\
		Dictionary support & Full
	\end{tabular}
\end{table}
There is conversion script for the English CMU
dictionary\footnote{\url{http://www.speech.cs.cmu.edu/cgi-bin/cmudict}} located
in \texttt{./par.eng/cmu2praatalign.py} that converts the CMU dictionary to
praatalign format. The scripts is a Python script and should download the
dictionary if you haven't done it yourself and will write it to a
\texttt{dict.eng} file by default. The usage is: \texttt{python
cmu2praatalign.py [inputfile [outputfile]]}.

\subsection{Sampa}
This setting is not really a language but allows the user to map any language
on the available phones found in Section~\ref{sec:slsampa}.
\begin{table}[H]
	\caption{Sampa language properties}
	\begin{tabular}{ll}
		Phonetizer support & None\\
		Ruleset support & Full\\
		Dictionary support & Full
	\end{tabular}
\end{table}

\subsection{Tzeltal}
Tzeltal uses the general SAMPA models and information about the phones can be
your in Section~\ref{sec:slsampa}. Further specification:
\begin{table}[H]
	\caption{Tzeltal language properties}
	\begin{tabular}{ll}
		Phonetizer support & Full\\
		Ruleset support & Full\\
		Dictionary support & Full
	\end{tabular}
\end{table}

\subsection{Spanish}
A detailed description of the Spanish phones can be found in
Section~\ref{sec:slspanish}. Further specification:
\begin{table}[H]
	\caption{Spanish language properties}
	\begin{tabular}{ll}
		Phonetizer support & Full\\
		Ruleset support & Full\\
		Dictionary support & Full
	\end{tabular}
\end{table}

The phonetizer for spanish automatically reduces truncated words that are
marked with brackets. For example: \texttt{cuida(do)} is phonetized as
\texttt{cuida}. The spanish phonetizer also clears up some non speech
annotations. Such as \texttt{[b]} etc.

\section{Scriptability and batch processing}
\subsection{Non interactive settings file creation}
Although the praatalign script is inherently interactive it is still possible
to batch process corpora using simple praat scripts. To facilitate this
function a file called \texttt{\$DIR/settings\_ni.praat} can be run where
\texttt{\$DIR} is the location of the plugin files. The location of the plugin
files for your operating system can be found in Section~\ref{sec:installation}
in the manual installation section. The \texttt{settings\_ni.praat} is a
stripped down version of the settings dialog present in the aligner. Since it
is using a praat form to ask for the user input, in contrary to the pause
dialog in the normal settings scirpt, it can be run non interactively by
running the script from a praat script.

For example if you want to setup the aligner to align a tzeltal file with all
custom values on linux under the user frobnicator you can put this in your
script to setup the aligner:

\begin{lstlisting}
runScript: "/home/frobnicator/.praat-dir/plugin_pralign/settings_ni.praat",
..."custom_phone_tier", "custom_word_tier", "tze", "/some/path/to/dict",
..."/some/path/to/ruleset", 0, "/some/path/to/logfile",
..."/usr/bin/sox", "/usr/bin/HVite", "/usr/bin/HCopy"
\end{lstlisting}

Note that due to the lack of interactivity the format is a little bit
different. The differences are:
\begin{itemize}
	\item \texttt{lan} must be the three letter languag code.
	\item \texttt{dic} and \texttt{rul} must be the full path, when you do not
want to use a dictionary or ruleset you must use \texttt{None}.
	\item \texttt{sox}, \texttt{hvb} and \texttt{hcb} must be the full path, when
you do not want a custom location you must use respectively \texttt{sox},
\texttt{HVite} and \texttt{HCopy}
\end{itemize}

\subsection{Example}
When you then open a \texttt{TextGrid} and a \texttt{LongSound} file and do
\texttt{View \& Edit} to open the editor you can run the alignment from the
script using the button text as function. For example the script could look
like the script in Listing~\ref{lis:scriptab}.

\begin{lstlisting}[caption={Example scriptability},label={lis:scriptab}]
# We assume the LongSound and TextGrid are selected previously

# Generate the settings file
	runScript: "/home/frobnicator/.praat-dir/plugin_pralign/settings_ni.praat",
..."phontier", "wordtier", "cantier", "llhtier", "spa", "/some/path/to/dict",
..."/some/path/to/ruleset", 0.05, "/some/path/to/logfile",
..."/usr/bin/sox", "/usr/bin/HVite", "/usr/bin/HCopy", "/usr/bin/python"

# Spawn the editor
View & Edit

# Open the editor
editor: "TextGrid " + objectname$
	# This bit of code is a small snippet to select a specific tier with the
	# index: tiernum, tiernum is obtained by querying all tiers outside the
	# editor and finding the tier that matches the name
	currenttiernum = -1
	while currenttiernum <> tiernum
		Select next tier
		inf$ = Editor info
		currenttiernum = extractNumber(inf$, "Selected tier: ")
	endwhile
	
	# Do the actual alignment
	Align current tier
	# When this is done aligned data can be found in: custom_phone_tier and 
	# custom_word_tier. $
endeditor
\end{lstlisting}

\section{Troubleshooting}
\begin{itemize}
	\item \textbf{Some words are ignored and thus not aligned}\\
		Some phonetizers phonetize unphonetizable words into an empty word to avoid
		throwing exceptions. When words are not phonetized it means that it is not
		in the dictionary nor phonetizable. To fix this you can edit the phonetizer
		or add the word to the dictionary.
	\item \textbf{SoX, HCopy or HVite not found}\\
		When this message pops up it means that the aligner does not have access to
		a \texttt{sox}, \texttt{HCopy} or \texttt{HVite} executable. Note that by
		default the aligner only looks in the current \texttt{PATH} and that it
		does not source any bashrc or profile file. To fix this you can add the
		exact path of the executables in the settings menu.
	\item \textbf{Log shows: }\texttt{sox FAIL trim: Position 1 is behind the
following position.}
		You are aligning an annotation that lies outside the wave file.
	\item \textbf{Other errors}\\
		When the plugin crashes without any reason you should enable logging in the
		settings menu to see where it crashes. If the problem is not solvable
		please file a bugreport via
		github~\footnote{\url{https://github.com/dopefishh/praatalign/issues}} or
		contact us directly via e-mail.
\end{itemize}


\chapter{Extending praatalign}
\section{Introduction}
Extending the aligner with new languages should be very easy for languages that
can be mapped on the current SAMPA model or on any other existing model(maus
model).
Adding a language with a new model could be possible but no support will be
given, however you can always try, you can even try getting help. Adding a
language requires a couple of components that need to be written or adapted.

\section{Phonetizer}
Phonetization of your language is the most elegant solution of translating the
graphemes to phonemes. Implementing a phonetizer is as easy as implementing one
function called \texttt{phonetizeword}. A skeleton class can be found in
\texttt{phonetizer.py}. The function in the skeleton class is accompanied by
comments. A phonetized utterance is always of the following form:\\
\texttt{utt=[word1, word2, \ldots, wordn]}, 
\texttt{word=[pron1, pron2, \ldots, pronn]} and
\texttt{pron=[phone1, phone2, \ldots, phonen]} and every phone is a string.
So if you want to use the skeleton class with the \texttt{phonetizeword}
function you need to return a list of lists of strings where every
string is a phone from the model. If you also want to do utterance based
translation you need to return a list of lists of lists of strings.

\section{Dictionary}
If you do not want to use a phonetizer you can also suffice with only using a
dictionary based translation. Dictionary based translation still needs to be
loaded as a phonetizer though. All phonetizers include also a dictionary based
lookup. In the \texttt{phonetizer.py} a dictionary phonetizer is already
present called \texttt{PhonetizerDictionary}. There is also a loopback
phonetizer that takes the literal annotation as transcription. This phonetizer
is currently not used but could be used when an exact phonetic translation is
already available.

\section{Adding the language to the aligner}
When you have the translation from grapheme to phoneme the only thing that
needs to be done is adding it to the script files.
\begin{itemize}
	\item \texttt{phonetizer.py}\\
		On the bottom of this file there is a dictionary containing all the
		translations from language code to phonetizer and parameters directory. You
		need to add your language to that dictionary.
	\item \texttt{settings.praat}\\
		In this file you need to add stuff on multiple locations, namely within the
		second \texttt{if} that relies in the first outer \texttt{if} block you
		need to add your language with its appropriate position. When you want your
		language on top you need to adapt the other numbers too.

		Finally within the \texttt{pause} block you need to add your language code
		in the \texttt{optionMenu:} block on the same position as specified
		earlier.
\end{itemize}
When you have changed these files properly your language should be available in
the menus and work out of the box.

\chapter{Appendices}
\section{How to cite}
\begin{lstlisting}[caption={Bibtex snippet},breaklines=true]
@misc{praatalign,
	author={Lubbers, Mart and Torreira, Francisco},
	title={Praatalign: an interactive Praat plug-in for performing phonetic forced alignment},
	howpublished={\url{https://github.com/dopefishh/praatalign}},
	year={2013-2015},
	note={Version 1.6}
}
\end{lstlisting}

\newpage
\section{Dutch phone specification}
\label{sec:sldutch}
Mapping with SAMPA
alphabet\footnote{\url{http://www.phon.ucl.ac.uk/home/sampa/dutch.htm}}

\paragraph{Consonants}
\subparagraph{Plosives}\strut\\
\begin{tabular}{llll}
	Praatalign & SAMPA & Word & Praatalign Transcription\\
	\hline
		p & p & pak & p A k\\
		b & b & bak & b A k\\
		t & t & tak & t A k\\
		d & d & dak & d A k\\
		k & k & kap & k A p\\
		g & g & goal & g o: l
\end{tabular}

\subparagraph{Fricatives}\strut\\
\begin{tabular}{llll}
	Praatalign & SAMPA & Word & Praatalign Transcription\\
	\hline
	f & f & fel & f E l\\
	v & v & vel & v E l\\
	s & s & sein & s E i n\\
	z & z & zijn & z E i n\\
	x & x & toch & t o x\\
	G & G & goed & G u t\\
	h & h & hand & h A n t\\
	Z & Z & bagage & b A g a: Z @\\
	S & S & show & S o: u
\end{tabular}

\subparagraph{Sonorants}\strut\\
\begin{tabular}{llll}
	Praatalign & SAMPA & Word & Praatalign Transcription\\
	\hline
	m & m & met & m E t\\
	n & n & net & n E t\\
	N & N & bang & b A N\\
	l & l & land & l A n t\\
	r & r & rand & r A n t\\
	w & w & wit & w I t\\
	j & j & ja & j a:
\end{tabular}

\paragraph{Vowels}
\subparagraph{Checked}\strut\\
\begin{tabular}{llll}
	Praatalign & SAMPA & Word & Praatalign Transcription\\
	\hline
	I & I & pit & p I t\\
	E & E & pet & p E t\\
	A & A & pat & p A t\\
	O & O & pot & p O t\\
	Y & Y & put & p Y t\\
	@ & @ & gemakkelijk & G @ m A k @ l @ k
\end{tabular}

\subparagraph{Free}\strut\\
\begin{tabular}{llll}
	Praatalign & SAMPA & Word & Praatalign Transcription\\
	\hline
	i & i & vier & v i r\\
	y & y & vuur & v y r\\
	u & u & voer & v u r\\
	a: & a: & naam & n a: m\\
	e: & e: & veer & v e: r\\
	P2: & 2: & deur & d P2: r\\
	o: & o: & voor & v o: r\\
	EI & Ei & fijn & f EI n\\
	P9y & 9y & huis & h P9y s\\
	Au & Au & goud & x Au t
\end{tabular}

\subparagraph{Diphthongs}\strut\\
\begin{tabular}{llll}
	Praatalign & SAMPA & Word & Praatalign Transcription\\
	\hline
	a:i & a:i & draai & d r a: i\\
	o:i & o:i & mooi & m o: i\\
	ui & ui & roeiboot & r ui b o: t\\
	iu & iu & nieuw & n iu\\
	yu & yu & duw & d yu\\
	e:u & e:u & sneeuw & s n e: u
\end{tabular}

\subparagraph{Marginals}\strut\\
\begin{tabular}{llll}
	Praatalign & SAMPA & Word & Praatalign Transcription\\
	\hline
	E: & E: & cr\'eme & k r E: m\\
	P9: & 9: & freule & f r P9: l @\\
	O: & O: & roze & r O: z @
\end{tabular}

\newpage
\section{English phone specification}
\label{sec:slenglish}
\begin{longtable}{llp{0.3\textwidth}ll}
	Praatalign & SAMPA & Phonetics & Examples & ISO639-3\\
	\hline
	\textless usb\textgreater & & human noise, garbage & & xxx\\
	\textless nib\textgreater & & noise, non-human & & xxx\\
	\textless p:\textgreater & & silence interval & & xxx\\
	a\textasciitilde & a\textasciitilde & nasalized central open vowel & vent & fra\\
	E\textasciitilde & E\textasciitilde & nasalized lengthened front half open unrounded vowel & & deu\\
	o\textasciitilde & o\textasciitilde & nasalized back half closed rounded vowel & bon & fra\\
	3: & 3: & lengthened front half open unrounded vowel & furs & eng\\
	i: & i: & lengthened front closed unrounded vowel & mieten & deu\\
	6: & 6: & lengthened central neutral unrounded vowel & & aus\\
	\}: & \}: & lengthened central closed rounded vowel & pool & aus\\
	e: & e: & lengthened front half closed unrounded vowel & mehr & deu\\
	o: & o: & lengthened back half closed rounded vowel & Sohle & deu\\
	u: & u: & lengthened back closed rounded vowel & Hut & deu\\
	A: & A: & lengthened back open unrounded vowel & stars & eng\\
	O: & O: & lengthened back half open rounded vowel & cause & eng\\
	@\} & @\} & diphthong & & eng\\
	Ae & Ae & diphthong & & aus\\
	\{I & \{I & diphthong & & aus\\
	\{O & \{O & diphthong & & aus\\
	oI & oI & diphthong & & aus\\
	eI & eI & diphthong & raise & eng\\
	aI & aI & diphthong & Bein & deu\\
	OI & OI & diphthong & noise & eng\\
	@U & @U & diphthong & nose & eng\\
	aU & aU & diphthong & Haus & deu\\
	I@ & I@ & diphthong & dears & eng\\
	e@ & e@ & diphthong & stairs & eng\\
	U@ & U@ & diphthong & cures & eng\\
	tS & tS & voiceless postalveolar affricate & English chair & xxx\\
	dZ & dZ & voiced postalveolar affricate & English gin & xxx\\
	e & e & front half closed unrounded vowel & US English bear & xxx\\
	\{ & \{ & front open unrounded vowel & English cat & xxx\\
	Q & Q & back open rounded vowel & British English not, cough & xxx\\
	O & O & back half open rounded vowel & British English law & xxx\\
	V & V & back half open unrounded vowel & RP and US English run & xxx\\
	U & U & back closed rounded vowel somewhat more centralised and relaxed & English put, Buddhist & xxx\\
	@ & @ & central neutral unrounded vowel & English about, winner & xxx\\
	i & i & front closed unrounded vowel & English see & xxx\\
	u & u & back closed rounded vowel & English soon & xxx\\
	o & o & back half closed rounded vowel & US English sore & xxx\\
	E & E & front half open unrounded vowel & English bed & xxx\\
	6 & 6 & central neutral unrounded vowel & German besser & xxx\\
	p & p & voiceless bilabial plosive & English pen & xxx\\
	b & b & voiced bilabial plosive & English but & xxx\\
	t & t & voiceless alveolar plosive & English two & xxx\\
	d & d & voiced alveolar plosive & English do & xxx\\
	k & k & voiceless velar plosive & English skill & xxx\\
	g & g & voiced velar plosive & English go & xxx\\
	f & f & voiceless labiodental fricative & English fool & xxx\\
	v & v & voiced labiodental fricative & English voice & xxx\\
	T & T & voiceless dental fricative & English thing & xxx\\
	D & D & voiced dental fricative & English this & xxx\\
	s & s & voiceless alveolar fricative & English see & xxx\\
	z & z & voiced alveolar fricative & English zoo & xxx\\
	S & S & voiceless postalveolar fricative & English she & xxx\\
	Z & Z & voiced postalveolar fricative & English pleasure & xxx\\
	h & h & voiceless glottal fricative & English ham & xxx\\
	m & m & bilabial nasal & English man & xxx\\
	n & n & alveolar nasal & English no & xxx\\
	N & N & velar nasal & English ring & xxx\\
	r & r & alveolar trill & Spanish perro & xxx\\
	l & l & alveolar lateral approximant & English left & xxx\\
	w & w & labial-velar approximant & English we & xxx\\
	j & j & palatal approximant & English yes & xxx\\
	I & I & front closed unrounded vowel, but somewhat more centralised and relaxed, in Polish: mid closed unrounded & English city & xxx\\
	? & ? & glottal stop & German Verein & xxx\\
	x & x & voiceless velar fricative & Scots loch & xxx\\
	C & C & voiceless palatal fricative & German Ich & xxx\\
	W & W & voiceless labial-velar fricative & & xxx\\
	\textless & & recording initial silence & & xxx\\
	\textgreater & & recording trailing silence & & xxx\\
	\# & & inter-word silence & & xxx\\
\end{longtable}

\newpage
\section{SAMPA phone specification}
\label{sec:slsampa}
\begin{longtable}{llp{0.6\textwidth}ll}
%\caption{SAMPA phone specification table}
	Praatalign & SAMPA & Phonetics & Example & ISO6393-9\\
	\hline
	i: & i: & lengthened front closed unrounded vowel & mieten & deu\\
	ii & ii & lengthened front closed unrounded vowel & riisu & ekk\\
	e: & e: & lengthened front half closed unrounded vowel & mehr & deu\\
	ee & ee & lengthened front half closed unrounded vowel & keere & ekk\\
	E: & E: & lengthened front half open unrounded vowel & Mär & deu\\
	y: & y: & lengthened front closed rounded vowel & Tür & deu\\
	2: & 2: & lengthened front half closed rounded vowel & Höhle & deu\\
	a: & a: & lengthened central open vowel & Haar & deu\\
	u: & u: & lengthened back closed rounded vowel & Hut & deu\\
	o: & o: & lengthened back half closed rounded vowel & Sohle & deu\\
	3: & 3: & lengthened front half open unrounded vowel & furs & eng\\
	A: & A: & lengthened back open unrounded vowel & stars & eng\\
	O: & O: & lengthened back half open rounded vowel & cause & eng\\
	6: & 6: & lengthened central neutral unrounded vowel & & aus\\
	\}: & \}: & lengthened central closed rounded vowel & pool & aus\\
	9: & 9: & lengthened front half open rounded vowel & & nld\\
	\{\{ & \{\{ & lengthened front open unrounded vowel & kääru & ekk\\
	yy & yy & lengthened front closed rounded vowel & müüri & ekk\\
	22 & 22 & lengthened front half closed rounded vowel & nööri & ekk\\
	uu & uu & lengthened back closed rounded vowel & kuuri & ekk\\
	oo & oo & lengthened back half closed rounded vowel & poori & ekk\\
	77 & 77 & back half closed unrounded vowel & sõõre & ekk\\
	AA & AA & lengthened back open unrounded vowel & vaaru & ekk\\
	aU & aU & diphthong & Haus & deu\\
	aI & aI & diphthong & Bein & deu\\
	ai & ai & diphthong & & ita\\
	a:i & a:i & diphthong & & nld\\
	Ae & Ae & diphthong & & aus\\
	Au & Au & diphthong & & nld\\
	OY & OY & diphthong & heulen & deu\\
	eI & eI & diphthong & raise & eng\\
	e@ & e@ & diphthong & stairs & eng\\
	ei & ei & diphthong & & ita\\
	e:i & e:i & diphthong & & nld\\
	eU & eU & diphthong & & por\\
	EI & EI & diphthong & raise & eng\\
	Ei & Ei & diphthong & & ita\\
	\{I & \{I & diphthong & & aus\\
	\{O & \{O & diphthong & & aus\\
	I@ & I@ & diphthong & dears & eng\\
	Ii: & Ii: & diphthong & accede & aus\\
	i:@ & i:@ & diphthong & memorial & nze\\
	io & io & diphthong & & ita\\
	iu & iu & diphthong & & nld\\
	ja & ja & diphthong & & ita\\
	jo & jo & diphthong & & ita\\
	ju & ju & diphthong & & ita\\
	oI & oI & diphthong & & aus\\
	oi & oi & diphthong & & ita\\
	o:i & o:i & diphthong & & nld\\
	oU & oU & diphthong & & por\\
	oE & oE & diphthong & & ita\\
	OI & OI & diphthong & noise & eng\\
	Oi & Oi & diphthong & & ita\\
	ue & ue & diphthong & & ita\\
	ui & ui & diphthong & & nld\\
	U@ & U@ & diphthong & cures & eng\\
	wa & wa & diphthong & & ita\\
	we & we & diphthong & & ita\\
	wi & wi & diphthong & & ita\\
	wO & wO & diphthong & & ita\\
	yu & yu & diphthong & & nld\\
	@U & @U & diphthong & nose & eng\\
	@\} & @\} & diphthong & & eng\\
	9y & 9y & diphthong & & nld\\
	QI & QI & diphthong & abide & aus\\
	U\} & U\} & diphthong & abuse & aus\\
	VU & VU & diphthong & acetone & aus\\
	Vi & Vi & diphthong & abased & aus\\
	\{o & \{o & diphthong & accounts & aus\\
	2i & 2i & diphthong & & ekk\\
	2i: & 2i: & diphthong & & ekk\\
	7o: & 7o: & diphthong & & ekk\\
	7u: & 7u: & diphthong & & ekk\\
	7u & 7u & diphthong & & ekk\\
	Ai & Ai & diphthong & & ekk\\
	Ai: & Ai: & diphthong & & ekk\\
	Ao: & Ao: & diphthong & & ekk\\
	Au: & Au: & diphthong & & ekk\\
	Ae: & Ae: & diphthong & & ekk\\
	ei: & ei: & diphthong & & ekk\\
	e:u & e:u & diphthong & & nld\\
	i\textasciitilde & i\textasciitilde & nasalized front closed unrounded vowel & & xxx\\
	e\textasciitilde & e\textasciitilde & nasalized front half closed unrounded vowel & vin & fra\\
	a\textasciitilde & a\textasciitilde & nasalized central open vowel & vent & fra\\
	o\textasciitilde & o\textasciitilde & nasalized back half closed rounded vowel & bon & fra\\
	9\textasciitilde & 9\textasciitilde & nasalized front half open rounded vowel & neuv & fra\\
	E\textasciitilde & E\textasciitilde & nasalized lengthened front half open unrounded vowel & & deu\\
	O\textasciitilde & O\textasciitilde & nasalized back half open rounded vowel & & deu\\
	u\textasciitilde & u\textasciitilde & nasalized back closed rounded vowel & & xxx\\
	a:\textasciitilde & a:\textasciitilde & nasalized lengthened central open vowel & & deu\\
	E:\textasciitilde & E:\textasciitilde & nasalized lengthened front half open unrounded vowel & & deu\\
	o:\textasciitilde & o:\textasciitilde & nasalized lengthened back half closed rounded vowel & & deu\\
	O:\textasciitilde & O\textasciitilde & nasalized lengthened back half open rounded vowel & & nld\\
	ts\_j & ts' & palatalized voiceless alveolar affricate & c'ma & pol\\
	dz\_j & dz' & palatalized voiced alveolar affricate & dz'wig & pol\\
	s\_j & s' & palatalized voiceless alveolar fricative & syk & pol\\
	z\_j & z' & palatalized voiced alveolar fricative & zbir & pol\\
	n\_j & n' & palatalized alveolar nasal & kon' & pol\\
	l\_j & l' & palatalized alveolar lateral approximant & pali & ekk\\
	t\_j & t' & palatalized voiceless alveolar plosive & padi & ekk\\
	d\_j & d' & voiced alveolar plosive English & gyár & hun\\
	g\_j & g' & palatalized voiced velar plosive & Gienek & pol\\
	x\_j & x' & palatalized voiceless velar fricative & hiacynt & pol\\
	k\_j & k' & palatalized voiceless velar plosive & kierowca & pol\\
	p\_j & p' & palatalized voiceless bilabial plosive & piasek & pol\\
	tt\_j & t't & palatalized long voiceless alveolar plosive & pati & ekk\\
	ss\_j & s's & palatalized long voiceless alveolar fricative & kassi & ekk\\
	nn\_j & n'n & palatalized long alveolar nasal & panni & ekk\\
	ll\_j & l'l & palatalized alveolar lateral approximant & palli & ekk\\
	p\_h & p\_h & aspirated voiceless bilabial plosive & & spa\\
	t\_h & t\_h & aspirated voiceless alveolar plosive & & spa\\
	k\_h & k\_h & aspirated voiceless velar plosive & & spa\\
	tt & tt & geminate of t & fatto & ita\\
	pp & pp & geminate of p & & ita\\
	kk & kk & geminate of k & & ita\\
	dd & dd & geminate of d & & ita\\
	gg & gg & geminate of g & & ita\\
	bb & bb & geminate of b & & ita\\
	ttS & ttS & geminate of tS & & ita\\
	tts & tts & geminate of ts & & ita\\
	ddZ & ddZ & geminate of dZ & & ita\\
	ddz & ddz & geminate of dz & zona & ita\\
	vv & vv & geminate of v & & ita\\
	ss & ss & geminate of s & & ita\\
	SS & SS & geminate of S & & ita\\
	rr & rr & geminate of r & & ita\\
	nn & n & geminate of n & & ita\\
	mm & mm & geminate of m & & ita\\
	LL & LL & geminate of L & & ita\\
	ll & ll & geminate of l & & ita\\
	JJ & JJ & geminate of J & & ita\\
	jj & jj & geminate of j & & ekk\\
	ff & ff & geminate of f & & ita\\
	hh & hh & geminate of h & & ekk\\
	ttS\_cl & & closure of ttS & & ita\\
	ttS\_rl & & release of ttS & & ita\\
	tts\_cl & & closure of tts & & ita\\
	tts\_rl & & release of tts & & ita\\
	ddZ\_cl & & closure of ddZ & & ita\\
	ddZ\_rl & & release of ddZ & & ita\\
	ddz\_cl & & closure of ddz & zona & ita\\
	ddz\_rl & & release of ddz & zona & ita\\
	tS\_cl & & closure of tS & & ita\\
	tS\_rl & & release of tS & & ita\\
	ts\_cl & & closure of ts & & ita\\
	ts\_rl & & release of ts & & ita\\
	dZ\_cl & & closure of dZ & & ita\\
	dZ\_rl & & release of dZ & & ita\\
	dz\_cl & & closure of dz & & ita\\
	dz\_rl & & release of dz & & ita\\
	tt\_rl & & release of tt & & ita\\
	tt\_cl & & closure of tt & & ita\\
	pp\_cl & & closure of pp & & ita\\
	pp\_rl & & release of pp & & ita\\
	kk\_cl & & closure of kk & & ita\\
	kk\_rl & & release of kk & & ita\\
	dd\_cl & & release of dd & & ita\\
	dd\_rl & & release of dd & & ita\\
	gg\_cl & & release of gg & & ita\\
	gg\_rl & & release of gg & & ita\\
	bb\_cl & & release of bb & & ita\\
	bb\_rl & & release of bb & & ita\\
	t\_cl & & closure of t & & ita\\
	t\_rl & & release of t & & ita\\
	p\_cl & & closure of p & & ita\\
	p\_rl & & release of p & & ita\\
	k\_cl & & closure of k & & ita\\
	k\_rl & & release of k & & ita\\
	g\_cl & & closure of g & & ita\\
	g\_rl & & release of g & & ita\\
	d\_cl & & closure of d & & ita\\
	d\_rl & & release of d & & ita\\
	b\_cl & & closure of b & & ita\\
	b\_rl & & release of b & & ita\\
	\textless & & recording initial silence & & xxx\\
	\textgreater & & recording trailing silence & & xxx\\
	\# & & inter-word silence & & xxx\\
	\textless nib\textgreater & & noise, non-human & & xxx\\
	\textless p:\textgreater & & silence interval & & xxx\\
	\textless usb\textgreater & & human noise, garbage & & xxx\\
\end{longtable}

\newpage
\section{Spanish phone specification}
\label{sec:slspanish}
The spanish mapping is an exact mapping with
SAMPA\footnote{\url{http://www.phon.ucl.ac.uk/home/sampa/spanish.htm}}.

\paragraph{Consonants}
\subparagraph{Plosives}\strut\\
\begin{tabular}{lll}
	Symbol & Word & Transcription\\
	\hline
 p & padre & p a D r e\\
 b & vino & b i n o\\
 t & tomo & t o m o\\
 d & donde & d o n d e\\
 k & casa & k a s a\\
 g & gata & g a t a
\end{tabular}

\subparagraph{Affricatives}\strut\\
\begin{tabular}{lll}
	Symbol & Word & Transcription\\
	\hline
	tS & mucho & m u tSo \\
	jj & hielo & jj e l o
\end{tabular}

\subparagraph{Fricatives}\strut\\
\begin{tabular}{lll}
	Symbol & Word & Transcription\\
	\hline
	f & fácil & f a T i l\\
	B & cabra & k a B r a\\
	T & cinco & T i n k o\\
	D & nada & n a D a\\
	s & sala & s a l a \\
	x & mujer & m u x e r\\
	G & luego & l w e G o
\end{tabular}

\subparagraph{Nasals}\strut\\
\begin{tabular}{lll}
	Symbol & Word & Transcription\\
	\hline
	m & mismo & m i s m o\\
	n & nunca & n u n k a\\
	J & año & a J o
\end{tabular}

\subparagraph{Liquids}\strut\\
\begin{tabular}{lll}
	Symbol & Word & Transcription\\
	\hline
	l & lejos & l e x o s\\
	L & caballo & k a b a L o\\
	r & puro & p u r o\\
	rr & torre & t o rr e
\end{tabular}

\subparagraph{Semivowels}\strut\\
\begin{tabular}{lll}
	Symbol & Word & Transcription\\
	\hline
	j & rei & rr e j\\
	 & pie & p j e\\
	w & deuda & d e w D a\\
	 & muy & m w i
\end{tabular}

\paragraph{Vowels}\strut\\
\begin{tabular}{lll}
	Symbol & Word & Transcription\\
	\hline
	i & pico & p i k o\\
	e & pero & p e r o\\
	a & valle & b a L e\\
	o & toro & t o r o\\
	u & duro & d u r o
\end{tabular}

\newpage
\section{Version history}
\begin{longtable}{|p{0.22\linewidth}p{0.8\linewidth}|}
	\hline
	1.60 (2015-08-00) & \tabitem Fixed menu name from \texttt{force} to
\texttt{forced}.\\
		& \tabitem Fixed silence problem in experimental models.\\
		& \tabitem Added version to setup dialog.\\
		& \tabitem Added language for using the general sampa model without
			phonetizer.\\
		& \tabitem Created a better error when loading rulesets or dictionaries in
			non supported encodings.\\
		& \tabitem Changed the licence to MIT.\\
	\hline
	1.50 (2015-07-07) & \tabitem Spanish models merged in original master.\\
		& \tabitem Procedures in different file, thus cleaner code.\\
		& \tabitem Reinitialized repo.\\
		& \tabitem Faster book compilation.\\
	\hline
	1.40 (2015-07-01) & \tabitem  Added branch for own spanish models.\\
		& \tabitem Updated book about ruleset.\\
		& \tabitem Changed name in menu from \texttt{Setup\ldots} to 
			\texttt{Set up\dots}.\\
	\hline
	1.39 (2015-04-01) & \tabitem Fixed tier alignment.\\
	\hline
	1.3 (2015-03-25) & \tabitem Added option for custom python path.\\
	 & \tabitem Greatly simplified rulesets.\\
	\hline
	1.2 (2015-02-18) & \tabitem Simplified installation scripts.\\
		& \tabitem Fixed a unicode bug in generating dictionaries.\\
	\hline
	1.1 (2015-01-30) & \tabitem Implemented better error messaging.\\
		& \tabitem Simplified the python code into one file.\\
		& \tabitem Fixed a bug that leaded to a messed up view.\\
		& \tabitem Fixed small phonetization errors.\\
	\hline
	1.0 (2015-01-09) & \tabitem Converted the readme to a pdf.\\
		& \tabitem Speeded up the process by disabling pitch, intensity,
spectrum, pulses and formants while aligning, the settings do get restored
afterwards.\\
		& \tabitem Fixed a bug that leaded to a messed up view.\\
	\hline
	0.9a (2014-12-02) & \tabitem Small bugfix in dictionary generation fixed.\\
	\hline
	0.9 & \tabitem Cleaned up some parts of the readme.\\
		& \tabitem Added language specific information.\\
		& \tabitem Added english as language. Although there is no phonetizing
implemented.\\
		& \tabitem README.html better with light background for code blocks.\\
		& \tabitem Updated citing method with bibtex.\\
	\hline
	0.8 (2014-10-31) & \tabitem Removed all the binary folders.\\
		& \tabitem Made the binary finding interactive.\\
		& \tabitem Made all the file chooser dialogs interactive.\\
	\hline
	0.7 (2014-10-29) & \tabitem Added windows support.\\
		&	\tabitem Cleaned up documentation.\\
		& \tabitem Removed binaries due htk licence.\\
	\hline
	0.6 (2014-10-22) & \tabitem Refactored and cleaned up the source.\\
	\hline
	0.5a (2014-09-08) & \tabitem Added comments to source code(praat).\\
		& \tabitem Cleaned up source.\\
	\hline
	0.5 (2014-09-04) & \tabitem Fixed acronyms in spanish.\\
		& \tabitem Fixed cleaning with extended boundaries.\\
		& \tabitem Added rudimentary ruleset implementation.\\
	\hline
	0.4 (2014-08-29) & \tabitem Added option for enlarging the boundaries
automatically.\\
	\hline
	0.21 (2014-08-13) & \tabitem Settings split in non interactive and
interactive so that the interactive one reflects the current settings.\\
	\hline
	0.2 (2014-08-11) & \tabitem Better mac compatibility.\\
	\hline
	0.1a (2014-06-30) & \tabitem Tier alignment fixed.\\
		& \tabitem Readme for dutch.\\
	\hline
	0.08 (2014-04-29) & \tabitem Cleaned up some stuff.\\
		& \tabitem Added dutch.\\
		& \tabitem Readme for spanish and sampa.\\
	\hline
	0.07 (2014-04-28) & \tabitem Non interactive alignment implemented.\\
		& \tabitem Table of contents in readme.\\
	\hline
	0.06 (2014-04-25) & \tabitem Conversion to editor scripts.\\
	\hline
	0.05 (2014-04-03) & \tabitem Better readme.\\
		& \tabitem Functional program for linux.\\
	\hline
	0.04 (2014-04-03) & \tabitem Pronunciation variants implemented.\\
	\hline
	0.03 (2014-03-31) & \tabitem Aligner works in python.\\
	\hline
	0.02 (2014-03-27) & \tabitem Python script around aligner started.\\
		& \tabitem Phonetizer skeleton done.\\
	\hline
	\caption{Version history}
\end{longtable}

\end{document}
