%&book
\begin{document}
\cleardoublepage
\maketitle
\setcounter{page}{1}
\tableofcontents
\chapter{Introduction}
\section{Introduction}
Praatalign is a plug-in for Praat that can be used to do forced phonetic
alignment on speech signals and in particular free speech. Praatalign combines
the powerful HTK toolkit and the well trained MAUS models with the
interactivity and modularity of Praat to create an interactive, intuitive and
extendible application. Text in \texttt{monospace} means that the word is a
command, variable or value. Praatalign has the capability to work with
python-programmed phonetizers, practical orthography, dictionaries and
pronunciation rules.

\chapter{Installation}
\section{Preparation}
The installation of the program is very straightforward, however installing the
dependencies might not be on some systems. Some programs are not included in
the package due to licencing and environment compatibility but they do need to
be installed in order for the plug-in to work properly. All programs Praatalign
depends on are free and open source. The following list of programs need to be
installed with their installation instructions.
\begin{itemize}
	\item \textbf{Praat}\\
		Praat is a program that allows you to do phonetic analysis and annotations
		with a computer and Praatalign uses Praat to provide an interactive user
		interface to the annotated sound files.
		\begin{itemize}
			\item \textbf{Windows}\\
				You can download and install Praat with the instructions on
				\url{http://www.fon.hum.uva.nl/praat/download_win.html}.
			\item \textbf{Mac}\\
				You can download and install Praat with the instructions on
				\url{http://www.fon.hum.uva.nl/praat/download_mac.html}.
			\item \textbf{Linux and other *NIX}\\
				In most of the cases the standard package manager coming with the
				Linux distribution also Praat. However this is usually an old version
				so use with care. You can download and install the latest Praat with
				the instructions on
				\url{http://www.fon.hum.uva.nl/praat/download_linux.html}.
		\end{itemize}
	\item \textbf{Python}\\
		Python is used to interpret the scripts that run the core of the aligner.
		The script uses specifically Python 2. Python 3 is not supported. All
		scripts are tested with Python $2.7.x$, older version might work.
		\begin{itemize}
			\item \textbf{Windows}\\
				You can download and install from the Windows MSI installer found on
				this page under the Latest Python 2 Release link
				\url{https://www.python.org/downloads/windows/}.
			\item \textbf{Mac}\\
				Mac OS X 10.8 comes with Python $2.7$ preinstalled. If you run an
				obscure version without Python you can download it on 
				\url{https://www.python.org/downloads/mac-osx/}.
			\item \textbf{Linux and other *NIX}\\
				In almost all cases the standard package manager comes with a
				sufficiently up to date Python version and most of the distributions
				have Python preinstalled. If you work with an obscure system that does
				not have a package manager or the Python version in it is obselete you
				can download Python on \url{https://www.python.org/downloads/source/}.
		\end{itemize}
	\item \textbf{SoX}\\
		For processing the sound files in a very detailed and controlled way we
		use SoX. Although Praat also has sound processing capabilities SoX works
		better is some situations, this is because Praat does not allow you to
		specify certain options like sampling rate for all formats.
		\begin{itemize}
			\item \textbf{Windows}\\
				You can download SoX on
				\url{http://sourceforge.net/projects/sox/files/sox/}. Either download
				the executable and run it or unzip the zipfile.
			\item \textbf{Mac}\\
				You can download SoX on
				\url{http://sourceforge.net/projects/sox/files/sox/}. Download the
				zipfile and drag the contects to you \texttt{Applications}. Another
				option is to use homebrew and type \texttt{brew install sox} in a
				terminal.
			\item \textbf{Linux and other *NIX}\\
				In almost all cases the standard package manager comes with a
				sufficiently up to date SoX version. If you work with an obscure system
				that does not have a package manager or the SoX version in it is
				obselete it on 
				\url{http://sourceforge.net/projects/sox/files/sox/14.4.2/}
		\end{itemize}
	\item \textbf{HCopy \& HVite}\\
		HCopy and HVite are programs from the HTK toolkit and due to licencing
		issues we can not provide the binaries in a direct way. The program is for
		free but you are not allowed to distribute it.
		\begin{itemize}
			\item \textbf{Windows}\\
				You can download a zip file containing precompiled Windows executables
				on \url{http://htk.eng.cam.ac.uk/ftp/software/htk-3.3-windows-binary.zip}.
				Just unzip them and make sure to add them to \texttt{\%PATH\%} or to
				point Praatalign to it in the setup screen.
			\item \textbf{Mac, Linux and other *NIX}\\
				Installing HCopy and HVite is probably the hardest on Linux and Mac
				since you need to compile the binaries yourself. You can find the
				latest version on \url{http://htk.eng.cam.ac.uk/ftp/software/}.
				Download the zipfile, extract the zipfile and go to the folder with
				your terminal. While you can compile the entire toolkit, Praatalign is
				only interested in HCopy and HVite. Thus the following commands
				suffice:
				\begin{lstlisting}[language=bash]
$ ./configure --disable-hlmtools --disable-hslab
$ make -j4 htktools
				\end{lstlisting}
				When the compilation has succeeded you can either add the binary
				directory \texttt{HTKTools} to \texttt{\$PATH} or point Praatalign to
				the binaries in the setup screen.
		\end{itemize}
\end{itemize}

\section{Installation}
\label{sec:installation}
The installation of the plug-in is very easy but the method differs for
different systems.\footnote{The plugin is tested on Windows 7, Windows Server
2008 via Windows Terminal Services, Windows 10, Linux and Mac. Other versions
or other operating systems might also work but are untested.}

\subsection{Automated installation}
Run the installation script for your system.
\begin{itemize}
	\item \textbf{Mac, Linux or *NIX}: \texttt{install.sh}\\
		Depending on the operating system you either have to run the script from
		the terminal or double click it from some explorer like program. Running
		the script from the terminal is very easy and preferred. Just start a
		terminal program and type the full and exact path in the terminal and press
		enter. For example on a Mac this will something like be:\\
		\texttt{/Users/frobnicator/Downloads/praatalign/install.sh}.
	\item \textbf{Windows}: \texttt{install\_win.bat}\\
		This installation script can be double clicked from the explorer and it
		will install Praatalign. When the installation is finished you can press
		Enter to make the windows disappear.
\end{itemize}

\subsection{Manual installation}
Copy the contents of the root directory to(you have to create the directory if
it does not already exist):
\begin{itemize}
	\item \textbf{Linux and *NIX}\\
		\texttt{\$\{HOME\}/.praat-dir/plugin\_pralign/}.
	\item \textbf{Windows}\\
		\texttt{\%USERPROFILE\%\textbackslash Praat\textbackslash plugin\_pralign/}
	\item \textbf{Mac}\\
		\texttt{\$\{HOME\}/Library/Preferences/Praat Prefs/plugin\_pralign/}
\end{itemize}

\chapter{Documentation}
\section{General information}
With the Praatalign plugin you can currently align out of the box the data
using Spanish, English, Dutch or SAMPA acoustic models from
MAUS~\footnote{\url{http://www.bas.uni-muenchen.de/Bas/BasMAUS.html}} created
by Schiel et al.~\cite{schiel1999}. These models are already included in the
package. Other models from MAUS can be easily added and will be in the future.
If you like to have a language added please contact us.
Presets for Australian English, Estonian, German, Hungarian, Italian, New
Zealand English Polish and Portuguese are available at minimum.

Dictionary, ruleset, universal phonetizer all other files are, and should be,
encoded in \texttt{UTF-8}. To enforce this the plug-in changes the default
behaviour of Praat every time Praat loads to make sure Praats reading and
writing preferences are set to \texttt{UTF-8}.

When the plug-in is successfully installed several menu items are added in the
\texttt{TextGrid} editor under the \texttt{Interval} menu. The added
functionality only works when you are editing a \texttt{TextGrid} and a
\texttt{LongSound} or a \texttt{Sound}\footnote{Sound files are written to disk
prior to alignment, thus \texttt{Longsound} is preferred}. Currently the plugin
is only tested on \texttt{WAVE} files. It should however work on all sound
filetypes SoX can detect from the extension.

\section{Menu items}
Almost all menu items will fail when there is no settings file present. The
settings file has to be created by running \texttt{Set up forced
alignment\ldots} interactively or by running \texttt{settings\_ni.praat} in a
script.

\subsection{\texttt{Generate dictionary from tier}}
This functions allows the user to generate a dictionary containing all the
missing or unphonetizable words from the currently selected tier using the
current settings the plugin is initialized with. The plugin will prompt you
after pressing the button for a location for the dictionary file. When this
process is done the user can add the pronunciations after every entry that
is found in the skeleton dictionary. Note that there is no sanitation
applied on the words. This means that if the phonetizer removes punctuation
it can still be present in the dictionary.

\subsection{\texttt{Clean selection}}
This function is a helper function to clean up old or wrong alignment. When the
function runs all annotation data within the selection within the selected tier
will be removed. 

\emph{Note that this is not necessary to do before an alignment because this
function runs by default before any alignment.}

\subsection{\texttt{Align current interval}}
This function aligns the current selected interval on the current selected
tier. When selecting a small interval it should not take much time at all. When
you select an interval from an output tier(phone, word, canonical or log) the
function will prompt you to make sure this is what you want to do. This
functions first clears out the annotations on the output tiers.

\subsection{\texttt{Align current tier}}
This function aligns the entire selected tier. Aligning an entire tier can take
some time, especially when you have a lot of pronunciation variants. This
functions first clears out the annotations on the output tiers.

\subsection{\texttt{Set up forced alignment\ldots}}
This functions spawns two option menus that will create the necessary settings
file. When you finish the forms a settings file is written to disk.

\subsubsection{Basic options}	
The first form contains all the basic settings needed for alignment. It also
shows the version of the plugin on the first line. The following options can be
entered. All settings are analogous to the names of the settings in the
\texttt{settings\_ni.praat} script.
\begin{itemize}
	\item \texttt{new}: %
		Name of the output tier storing the phone level alignment

		In this option you specify the name of the tier where the phone level
		alignment is stored, this can be either an existing tier or a non existing
		tier. If the tier does not exist it will be created upon doing the first
		alignment. When you leave this field empty no phone level tier will be
		created.
	\item \texttt{wrd}: %
		Name of the output tier storing the word level alignment

		In this option you specify the name of the tier where the word level
		alignment is stored, this can be either an existing tier or a non existing
		tier. If the tier does not exist it will be created upon doing the first
		alignment. When you leave this field empty no word level tier will be
		created.
	\item \texttt{can}: %
		Name of the output tier storing the canonical pronunciation

		In this option you specify the name of the tier where the canonical
		pronunciation of every word is stored, this can be either an existing tier
		or a non existing tier. If the tier does not exist it will be created upon
		doing the first alignment. When you leave this field empty no canonical
		pronunciation tier will be created.
	\item \texttt{llh}: %
		Name of the output tier storing the log likelihood

		In this option you specify the name of the tier where the log likelihood of
		every phone is stored , this can be either an existing tier or a non
		existing tier. If the tier does not exist it will be created upon doing the
		first alignment. When you leave this field empty no log likelihood tier
		will be created.
	\item \texttt{model}: %
		Select model

		In this option you specify which acoustic model to use. More info about the
		models can be found in Section~\ref{sec:models}.
	\item \texttt{lan}: %
		Select model

		In this option you specify which phonetizer to use. More info about the
		models can be found in Section~\ref{sec:phonetizers}.

		\emph{Note that if you select the universal phonetizer you will be prompted
		to select the universal phonetizer file. More about the universal
		phonetizer in Section~\ref{sec:univphonetizer}}
	\item \texttt{dic}, \texttt{dictionary}: %
		Select a dictionary file

		In this option you can specify a dictionary file. When you tick the box you
		will be prompted to select a dictionary file. When a dictionary is already
		selected the \texttt{dictionary} option is added which contains the path to
		the file. When you want to switch to using no dictionary you can clear that
		box and leave the \texttt{dic} box unticked.
	\item \texttt{rul}, \texttt{ruleset}: %
		Select a ruleset file

		In this option you can specify a ruleset file. When you tick the box you
		will be prompted to select a ruleset file. When a dictionary is already
		selected the \texttt{ruleset} option is added which contains the path to
		the file. When you want to switch to using no ruleset you can clear that
		box and leave the \texttt{rul} box unticked.
	\item \texttt{thr}: %
		Set the size to add to the annotations

		In this option you can specify an extra margin used for every annotation.
		When the annotations are placed to close to the real sound the initial
		pause can clobber up the beginning of speech and that can reduce the
		performance. Setting the \texttt{thr} value to $0.1$ will for example increase
		all boundaries from annotations with $100$ms. Note that this does not
		change the original annotation and it will only increase the widen the
		annotation when there is room to do so, meaning that it will not create
		overlap with other annotations.
\end{itemize}

\subsubsection{Advanced options}	
The second form contains all the more advanced settings needed for alignment.
It shouldn't be necessary to change these options regularly. All settings are
analogous to the names of the settings in the \texttt{settings\_ni.praat}
script.

\begin{itemize}
	\item \texttt{log}: %
		Set a location for the logfile

		In this option you can specify a location to write a debug log to. When you
		want to switch to not using a logfile you can redirect the log to either
		\texttt{/dev/null} on Linux, Mac and other *NIX systems and \texttt{nul} on
		Windows.
	\item \texttt{sox}: %
		Set a SoX executable

		In this option you can specify a SoX executable. When you tick the box you
		will be prompted to select a SoX executable. When a SoX executable is
		already selected the \texttt{soxex} option is added which contains the path
		to the executable. When you want to switch to using the SoX executable in
		\texttt{PATH} you can clear that box leave the \texttt{sox} box unticked.

		\begin{itemize}
			\item \textbf{Windows}\\
				If you have installed sox using the MSI you can find \texttt{sox.exe}
				in \texttt{C:\textbackslash 
				Program Files (x86)\textbackslash sox-14-4-1} or
				\texttt{C:\textbackslash Program Files\textbackslash sox-14-4-1}.
				If you just downloaded the zip file you can just point to the location
				you extracted the archive and select \texttt{sox.exe}.
			\item \textbf{Mac}\\
				If you dragged sox to the \texttt{Applications} you can find it there
				and you can just point to the sox executable. If you installed sox via
				homebrew it is probably already in \texttt{\$PATH}. If this is not the
				case you can find the location by typing in a terminal \texttt{which
				sox} and pointing Praatalign to that location.
			\item \textbf{Linux and other *NIX}\\
				If you have installed sox using a package manager it probably already
				is in your \texttt{\$PATH}. If this is not the case you can find the
				location by typing in a terminal: \texttt{which sox} and pointing
				Praatalign to location.
		\end{itemize}
	\item \texttt{hvite}: %
		Set a HVite executable

		In this option you can specify a HVite executable. When you tick the box you
		will be prompted to select a HVite executable. When a HVite executable is
		already selected the \texttt{hviteex} option is added which contains the
		path to the executable. When you want to switch to using the HVite
		executable in \texttt{PATH} you can clear that box leave the \texttt{hvite}
		box unticked.

		\begin{itemize}
			\item \textbf{Windows}\\
				Point to the directory where you unzipped the file
				from HTK and select \texttt{HVite.exe}.
			\item \textbf{Mac, Linux, and other *NIX}\\
				Point to the directory where you compiled the tools from HTK and
				select \texttt{HVite}. It resides in the \texttt{HTKTools}
				directory.
		\end{itemize}
	\item \texttt{hcopy}: %
		Set a HCopy executable

		In this option you can specify a HCopy executable. When you tick the box you
		will be prompted to select a HCopy executable. When a HCopy executable is
		already selected the \texttt{hcopyex} option is added which contains the
		path to the executable. When you want to switch to using the HCopy
		executable in \texttt{PATH} you can clear that box leave the \texttt{hcopy}
		box unticked.

		\begin{itemize}
			\item \textbf{Windows}\\
				Point to the directory where you unzipped the file
				from HTK and select \texttt{HCopy.exe}.
			\item \textbf{Mac, Linux, and other *NIX}\\
				Point to the directory where you compiled the tools from HTK and
				select \texttt{HCopy}. It resides in the \texttt{HTKTools}
				directory.
		\end{itemize}
	\item \texttt{python}: %
		Set a Python executable

		In this option you can specify a Python executable. When you tick the box
		you will be prompted to select a Python executable. When a Python
		executable is already selected the \texttt{pythonex} option is added which
		contains the path to the executable. When you want to switch to using the
		Python executable in \texttt{PATH} you can clear that box leave the
		\texttt{python} box unticked.

		\begin{itemize}
			\item \textbf{Windows}\\
				Python can usually be found in \texttt{C:\textbackslash Python27}.
				From there you can select \texttt{python.exe}
			\item \textbf{Mac, Linux, and other *NIX}\\
				If you have installed sox using a package manager it probably
				already is in your \texttt{\$PATH}. If this is not the case you can
				find the location by typing in a terminal: \texttt{which python}
				and pointing Praatalign to that location. Note that in some systems
				\texttt{python} symlinks to \texttt{python3}, in that case point
				Praatalign to \texttt{python2}. If you still can not find the
				executable you can try using the search function in the file
				manager.
		\end{itemize}
\end{itemize}

\section{Dictionary}
To phonetize words Praatalign either uses the provided phonetizer or a
dictionary. Dictionaries are plain text files that contain words and one or
more pronunciations. A dictionary file is a \texttt{UTF-8} encoded file
containing non-empty lines separated by a newline character\footnote{On Mac,
Linux, and *NIX this is default, on Windows this can cause problems. When using
Praatalign on windows please refrain to a text editor that has newline
capabilities like Notepad++}. Lines starting with a \texttt{\#} will be ignored
and can thus be used as comments. The format of a dictionary entry is a word
followed by a tab followed by tab separated pronunciations. An example
dictionary can be found in Listing~\ref{lst:exampledictionary}

\begin{lstlisting}[caption={Example dictionary},label={lst:exampledictionary}]
# This is comment
# This is a word with two possible pronunciations
ado<TAB>a d o<TAB>a o
# These are words with one possible pronunciation
empatar<TAB>e m p a t a r
empataran<TAB>e m p a t a r a n
\end{lstlisting}
	
\section{Ruleset}
Besides generating pronunciation by using the dictionary and phonetization you
can also use rulesets to define pronunciation variants. Ruleset make you able
to define general rules applied over all words(phonetized words and dictionary
words). In this way you can easily define for example deletion rules.
A ruleset file is a \texttt{UTF-8} encoded file containing non-empty lines
separated by a newline character. Lines starting with a \texttt{\#} will be
ignored and can thus be used as comments. There are two ways of defining rules
for a ruleset.
\begin{itemize}
	\item \textbf{Simple}\\
		Simple rules are just find and replace queries. The first column is the
		target and the second column is the replace value. For example the deletion
		rule \texttt{a d o -> a o} can be written as \texttt{a d o<TAB>a o}.
	\item \textbf{Regular}\\
		Regular rules are like regular expressions and are therefore much more
		expressive but also more complicated to write. Regular rules start with a
		single tab character to denote that they are in fact regular expressions.
		For example a deletion rule that deletes a \texttt{d} between two vowels
		and potentially also across word boundaries can be written like \texttt{%
		<TAB>([aouie] \#?) d ([aouie])<TAB>\textbackslash 1 \textbackslash 2}.
		Internally we use the \texttt{re.sub} function from the Python \texttt{re}
		library\footnote{\url{https://docs.python.org/2/library/re.html}}. Besides
		that there are some extra shortcuts and options:
		\begin{itemize}
			\item \texttt{\textbackslash v} for vowels(\texttt{[aoeiu]})
			\item \texttt{\textbackslash c} for consonants(%
				\texttt{[\textasciicircum aoeiu]})
			\item \texttt{\#} for a inter word silence
		\end{itemize}
\end{itemize}


\section{Phonetizers}\label{sec:phonetizers}
\subsection{Spanish}
The Spanish phonetizer is designed only to work with the spanish models. It
removes a lot of non speech annotated symbols and does some tricks to get
exceptions well phonetized. It can be seen as an example of writing an advanced
phonetizer in Python.

\subsection{Tzeltal}
The Tzeltal phonetizer is an example of how to use the SAMPA models to align a
new language. Thus it only works for the SAMPA models. It removes some non
speech annotated symbols and does an almost literal character to character
translation.

\subsection{Universal}\label{sec:univphonetizer}
When you select the universal phonetizer you will be prompted to point the
plugin to an universal phonetizer file.
A universal phonetizer file is a \texttt{UTF-8} encoded file containing
non-empty lines separated by a newline character. Every line contains a
translation from practical orthography to phonetic transcription and the order
of appearance in the file is the order of importance in the phonetizer. When a
word gets phonetized the phonetizer tries to match the first rule in the
phonetizer file. For example see the start of an example file listed in
Listing~\ref{lst:exampleuniversalphonetizer} representing a translation from
spanish orthography to phonetic transcription. The safest order is always the
order in which the biggest sections are topmost in the file.

\begin{lstlisting}[caption={Example universal phonetizer file},
	label={lst:exampleuniversalphonetizer},
	literate={ü}{{\"u}}1 {ñ}{{\~n}}1 {ç}{{\c{c}}}1 {í}{{\'e}}1]
gue<TAB>g
gui<TAB>g
ch<TAB>t S
ce<TAB>T
ci<TAB>T
cí<TAB>T
gü<TAB>g u
ll<TAB>jj
qu<TAB>k
ñ<TAB>J
ç<TAB>T
j<TAB>x
c<TAB>k
v<TAB>b
w<TAB>b
z<TAB>T
y<TAB>j
q<TAB>k
...
\end{lstlisting}

\subsection{None}
The None phonetizer is a dummy phonetizer that does nothing. This means that
every word should be available in the dictionary.

\section{Scriptability and batch processing}
Note that this section is not updated as often as it should. Always check the
exact format in \texttt{settings\_ni.praat}.

\subsection{Non interactive settings file creation}
Although the Praatalign script is inherently interactive it is still possible
to batch process corpora using simple praat scripts. To facilitate this
function a file called \texttt{\$DIR/settings\_ni.praat} can be run where
\texttt{\$DIR} is the location of the plugin files. The location of the plugin
files for your operating system can be found in Section~\ref{sec:installation}
in the manual installation section. The \texttt{settings\_ni.praat} is a
stripped down version of the settings dialog present in the aligner. Since it
is using a praat form to ask for the user input, in contrary to the pause
dialog in the normal settings scirpt, it can be run non interactively by
running the script from a praat script.

For example if you want to setup the aligner to align a tzeltal file with all
custom values on linux under the user frobnicator you can put this in your
script to setup the aligner:

\begin{lstlisting}
runScript: "/home/frobnicator/.praat-dir/plugin_pralign/settings_ni.praat",
..."phon", "wrd", "can", "llh", "sampa", "tzeltal", "None", "/path/to/dict",
..."/path/to/ruleset", 0, "/some/path/to/logfile",
..."/usr/bin/sox", "/usr/bin/HVite", "/usr/bin/HCopy", "python"
\end{lstlisting}

Note that due to the lack of interactivity the format is a little bit
different. The differences are:
\begin{itemize}
	\item \texttt{lan} must be the language code as in the interactive settings.
	\item \texttt{mod} must be the model code as in the interactive settings.
	\item \texttt{pho} must be the path to the universal phonetizer. When you do
not want to use a universal phonetizer you must use \texttt{None}.
	\item \texttt{dic} and \texttt{rul} must be the full path, when you do not
want to use a dictionary or ruleset you must use \texttt{None}.
	\item \texttt{sox}, \texttt{hvb}, \texttt{hcb}, and \texttt{py} must be the
full path, when you do not want a custom location you must use respectively
\texttt{sox}, \texttt{HVite}, \texttt{HCopy} and \texttt{python}
\end{itemize}

\subsection{Example}
When you then open a \texttt{TextGrid} and a \texttt{LongSound} file and do
\texttt{View \& Edit} to open the editor you can run the alignment from the
script using the button text as function. For example the script could look
like the script in Listing~\ref{lis:scriptab}.

\begin{lstlisting}[caption={Example scriptability},label={lis:scriptab}]
# We assume the LongSound and TextGrid are selected previously

# Generate the settings file
	runScript: "/home/frobnicator/.praat-dir/plugin_pralign/settings_ni.praat",
..."phon", "wrd", "can", "llh", "sampa", "tzeltal", "None", "/path/to/dict",
..."/path/to/ruleset", 0, "/some/path/to/logfile",
..."/usr/bin/sox", "/usr/bin/HVite", "/usr/bin/HCopy", "python"

# Spawn the editor
View & Edit

# Open the editor
editor: "TextGrid " + objectname$
	# This bit of code is a small snippet to select a specific tier with the
	# index: tiernum, tiernum is obtained by querying all tiers outside the
	# editor and finding the tier that matches the name
	currenttiernum = -1
	while currenttiernum <> tiernum
		Select next tier
		inf$ = Editor info
		currenttiernum = extractNumber(inf$, "Selected tier: ")
	endwhile
	
	# Do the actual alignment
	Align current tier
	# When this is done aligned data can be found in: custom_phone_tier and 
	# custom_word_tier.
endeditor
\end{lstlisting}

\section{Troubleshooting}
\begin{itemize}
	\item \textbf{Some words are ignored and thus not aligned}\\
		Some phonetizers phonetize unphonetizable words into an empty word to avoid
		throwing exceptions. When words are not phonetized it means that it is not
		in the dictionary nor phonetizable. To fix this you can edit the phonetizer
		or add the word to the dictionary.
	\item Pop-up stating: \textbf{Error! check the text window for details...}\\
		This means something went wrong in the python script. Check the info
		window. Usually it is a missing binary, ruleset etc. It could also be
		that you used a phone that does not exist in the model
	\item Pop-up stating: \textbf{Unknown IO error}\\
		Some IO error occurred of an unknown type. Please check the logfile.
	\item Log stating: \texttt{sox FAIL trim: Position 1 is behind the
		following position.}
		You are aligning an annotation that lies outside the wave file.
	\item \textbf{Other errors}\\
		When the plugin crashes without any reason you should enable logging in the
		settings menu to see where it crashes. If the problem is not solvable
		please file a bugreport via
		github~\footnote{\url{https://github.com/dopefishh/praatalign/issues}} or
		contact us directly via e-mail.
\end{itemize}

\chapter{Extending Praatalign}
\section{Introduction}
Extending the aligner with new languages should be very easy for languages that
can be mapped on the current SAMPA model or on any other existing model(maus
model).
Adding a language with a new model could be possible but no support will be
given, however you can always try, you can even try getting help. Adding a
language requires a couple of components that need to be written or adapted.

\section{Phonetizer}
Phonetization of your language is the most elegant solution of translating the
graphemes to phonemes. Implementing a phonetizer is as easy as implementing one
function called \texttt{phonetizeword}. A skeleton class can be found in
\texttt{phonetizer.py}. The function in the skeleton class is accompanied by
comments. A phonetized utterance is always of the following form:\\
\texttt{utt=[word1, word2, \ldots, wordn]}, 
\texttt{word=[pron1, pron2, \ldots, pronn]} and
\texttt{pron=[phone1, phone2, \ldots, phonen]} and every phone is a string.
So if you want to use the skeleton class with the \texttt{phonetizeword}
function you need to return a list of lists of strings where every
string is a phone from the model. If you also want to do utterance based
translation you need to return a list of lists of lists of strings.

\section{Dictionary}
If you do not want to use a phonetizer you can also suffice with only using a
dictionary based translation. Dictionary based translation still needs to be
loaded as a phonetizer though. All phonetizers include also a dictionary based
lookup. In the \texttt{phonetizer.py} a dictionary phonetizer is already
present called \texttt{PhonetizerDictionary}. There is also a loopback
phonetizer that takes the literal annotation as transcription. This phonetizer
is currently not used but could be used when an exact phonetic translation is
already available.

\section{Adding the language to the aligner}
When you have the translation from grapheme to phoneme the only thing that
needs to be done is adding it to the script files.
\begin{itemize}
	\item \texttt{phonetizer.py}\\
		On the bottom of this file there is a dictionary containing all the
		translations from language code to phonetizer and parameters directory. You
		need to add your language to that dictionary.
	\item \texttt{settings.praat}\\
		In this file you need to add stuff on multiple locations, namely within the
		second \texttt{if} that relies in the first outer \texttt{if} block you
		need to add your language with its appropriate position. When you want your
		language on top you need to adapt the other numbers too.

		Finally within the \texttt{pause} block you need to add your language code
		in the \texttt{optionMenu:} block on the same position as specified
		earlier.
\end{itemize}
When you have changed these files properly your language should be available in
the menus and work out of the box.

\chapter{Appendices}
\section{How to cite}
\begin{lstlisting}[caption={Bibtex snippet}]
@misc{praatalign1.8,
	author={Lubbers, Mart and Torreira, Francisco},
	title={Praatalign: an interactive Praat plug-in for performing phonetic forced alignment},
	howpublished={\url{https://github.com/dopefishh/praatalign}},
	year={2013-2015},
	note={Version 1.8}
}
\end{lstlisting}

\section{Licence}
\lstinputlisting{../LICENCE}

\section{Acoustic model phone specifications}\label{sec:models}
\subsection{Spanish}
\label{sec:phspanish}
The Spanish mapping is an exact mapping with the spanish SAMPA
phoneset\footnote{\url{http://www.phon.ucl.ac.uk/home/sampa/spanish.htm}}.

{\small
\begin{longtable}{l|l|l|l}
	& Symbol & Word & Transcription\\
	\hline
	\multirow{6}{*}{Plosives} &
	p & padre & p a D r e\\
  & b & vino & b i n o\\
  & t & tomo & t o m o\\
  & d & donde & d o n d e\\
  & k & casa & k a s a\\
  & g & gata & g a t a\\
	\hline
	\multirow{2}{*}{Affricatives} &
	tS & mucho & m u tS o\\
	& jj & hielo & jj e l o\\
	\hline
	\multirow{7}{*}{Fricatives} &
	f & fácil & f a T i l\\
	& B & cabra & k a B r a\\
	& T & cinco & T i n k o\\
	& D & nada & n a D a\\
	& s & sala & s a l a \\
	& x & mujer & m u x e r\\
	& G & luego & l w e G o\\
	\hline
	\multirow{3}{*}{Nasals} &
	m & mismo & m i s m o\\
	& n & nunca & n u n k a\\
	& J & año & a J o\\
	\hline
	\multirow{4}{*}{Liquids} & 
	l & lejos & l e x o s\\
	& L & caballo & k a b a L o\\
	& r & puro & p u r o\\
	& rr & torre & t o rr e\\
	\hline
	\multirow{4}{*}{Semivowels} &
	j & rei & rr e j\\
	&  & pie & p j e\\
	& w & deuda & d e w D a\\
	&  & muy & m w i\\
	\hline
	\multirow{5}{*}{Vowels} &
	i & pico & p i k o\\
	& e & pero & p e r o\\
	& a & valle & b a L e\\
	& o & toro & t o r o\\
	& u & duro & d u r o\\
	\hline
	\multirow{4}{*}{Special} &
	\textless  & word initial silence & \\
	& \textgreater  & word final silence & \\
	& \# & inter word silence & \\
	& \textless nib\textgreater  & non speech sound & \\
	\hline
	\caption{Spanish phone specification}
\end{longtable}
}

\subsection{Dutch}
The Dutch mapping is an almost\footnote{Derivations are marked in bold} exact
mapping with the dutch SAMPA phoneset\footnote{\url{
http://www.phon.ucl.ac.uk/home/sampa/dutch.htm}}

{\small
\begin{longtable}{l|l|l|l}
	& Praatalign & Word & Transcription\\
	\hline
	\multirow{6}{*}{Plosives} &
	p & pak & p A k\\
	& b & bak & b A k\\
	& t & tak & t A k\\
	& d & dak & d A k\\
	& k & kap & k A p\\
	& g & goal & g o: l\\
	\hline
	\multirow{9}{*}{Fricatives} &
	f & fel & f E l\\
	& v & vel & v E l\\
	& s & sein & s E i n\\
	& z & zijn & z E i n\\
	& x & toch & t o x\\
	& G & goed & G u t\\
	& h & hand & h A n t\\
	& Z & bagage & b A g a: Z @\\
	& S & show & S o: u\\
	\hline
	\multirow{7}{*}{Sonorants} &
	m & met & m E t\\
	& n & net & n E t\\
	& N & bang & b A N\\
	& l & land & l A n t\\
	& r & rand & r A n t\\
	& w & wit & w I t\\
	& j & ja & j a:\\
	\hline
	\multirow{6}{*}{Checked vowels} &
	I & pit & p I t\\
	& E & pet & p E t\\
	& A & pat & p A t\\
	& O & pot & p O t\\
	& Y & put & p Y t\\
	& @ & gemakkelijk & G @ m A k @ l @ k\\
	\hline
	\multirow{10}{*}{Free vowels} &
	i & vier & v i r\\
	& y & vuur & v y r\\
	& u & voer & v u r\\
	& a: & naam & n a: m\\
	& e: & veer & v e: r\\
	& \textbf{P2:} & deur & d P2: r\\
	& o: & voor & v o: r\\
	& EI & fijn & f EI n\\
	& \textbf{P9y} &huis & h P9y s\\
	& Au & goud & x Au t\\
	\hline
	\multirow{6}{*}{Dipthongs} &
	a:i & draai & d r a: i\\
	& o:i &mooi & m o: i\\
	& ui & roeiboot & r ui b o: t\\
	& iu & nieuw & n iu\\
	& yu & duw & d yu\\
	& e:u & sneeuw & s n e: u\\
	\hline
	\multirow{3}{*}{Marginal vowels} &
	E: & cr\'eme & k r E: m\\
	& \textbf{P9:} & freule & f r P9: l @\\
	& O: & roze & r O: z @\\
	\hline
	\multirow{4}{*}{Special} &
	\textless & word initial silence & \\
	& \textgreater & word final silence & \\
	& \# & inter word silence & \\
	& \textless nib\textgreater & non speech sound & \\
	\hline
	\caption{Dutch phone specification}
\end{longtable}
}

\subsection{English}
The English mapping is an exact\footnote{Derivations are marked in bold} exact
mapping with the english SAMPA phoneset\footnote{\url{
https://www.phon.ucl.ac.uk/home/sampa/english.htm}} with some borrowed phones.

Also there is conversion script for the English CMU
dictionary\footnote{\url{http://www.speech.cs.cmu.edu/cgi-bin/cmudict}} located
in \texttt{./par.eng/cmu2praatalign.py} that converts the CMU dictionary to
Praatalign format. The scripts is a Python script and should download the
dictionary if you haven't done it yourself and will write it to a
\texttt{dict.eng} file by default. The usage is: \texttt{python
cmu2praatalign.py [inputfile [outputfile]]}.

{\small
\begin{longtable}{l|l|p{.3\linewidth}|l}
	& Symbol & Word & Transcription\\
	\hline
	\multirow{6}{*}{Plosives} &
	  p & pen &\\
	& b & but &\\
	& t & two &\\
	& d & do &\\
	& k & skill &\\
	& g & go &\\
	\hline
	\multirow{2}{*}{Affricatives} &
	tS & chair &\\
	& dZ & gin &\\
	\hline
	\multirow{9}{*}{Fricatives} &
		f & fool &\\
	& v & voice &\\
	& T & thing &\\
	& D & this &\\
	& s & sin &\\
	& z & zoo &\\
	& S & she &\\
	& Z & pleasure &\\
	& h & ham &\\
	\hline
	\multirow{3}{*}{Nasals} &
	m & man &\\
	& n & no &\\
	& N & ring &\\
	\hline
	\multirow{2}{*}{Liquids} &
	r & perro &\\
	& l & left &\\
	\hline
	\multirow{2}{*}{Sonorant glides} &
	w & we &\\
	& j & yes &\\
	\hline
	\multirow{6}{*}{Checked vowels} &
	I & English city &\\
	& e & bear &\\
	& \{ & cat &\\
	& Q & cough &\\
	& V & run &\\
	& U & put &\\
	\hline
	\multirow{1}{*}{Short vowels} &
	@ & about &\\
	\hline
	\multirow{13}{*}{Free vowels} &
	i: & ease &\\
	& eI & raise &\\
	& aI & rise &\\
	& OI & noise &\\
	& u: & lose &\\
	& @U & nose &\\
	& aU & rouse &\\
	& 3: & furs &\\
	& A: & stars &\\
	& O: & cause &\\
	& I@ & dears &\\
	& e@ & stairs &\\
	& U@ & cures &\\
	\hline
	\multirow{3}{*}{Special} &
	\textless & word initial silence & \\
	& \textgreater & word final silence & \\
	& \# & inter word silence & \\
	& \textless nib\textgreater & non speech sound & \\
	\hline\hline
	\multirow{22}{*}{Borrowed phones} & Symbol & description &
		example(language)\\
	\hline
	& a\textasciitilde & 
		nasalized central open vowel & vent(fra)\\
	& E\textasciitilde & 
		nasalized lengthened front half open unrounded vowel & (deu)\\
	& o\textasciitilde & 
		nasalized back half closed rounded vowel & bon(fra)\\
	& 6: & lengthened central neutral unrounded vowel & (aus)\\
	& \}: & lengthened central closed rounded vowel & pool(aus)\\
	& e: & lengthened front half closed unrounded vowel & mehr(deu)\\
	& o: & lengthened back half closed rounded vowel & Sohle(deu)\\
	& @\} & diphthong &\\
	& Ae & diphthong & (aus)\\
	& \{I & diphthong & (aus)\\
	& \{O & diphthong & (aus)\\
	& oI & diphthong & (aus)\\
	& O & back half open rounded vowel & law(brit)\\
	& i & front closed unrounded vowel & see\\
	& u & back closed rounded vowel & soon\\
	& o & back half closed rounded vowel & sore(us)\\
	& E & front half open unrounded vowel & bed\\
	& 6 & central neutral unrounded vowel & besser(deu)\\
	& ? & glottal stop & Verein(ger)\\
	& x & voiceless velar fricative & loch(scot)\\
	& C & voiceless palatal fricative & Ich(ger)\\
	& W & voiceless labial-velar fricative &\\
	\hline
	\caption{English phone specification}
\end{longtable}
}

\subsection{SAMPA}
{\small
\begin{longtable}{l|p{.3\linewidth}|p{.15\linewidth}|l|p{.15\linewidth}|l}
	Sym & description & examples & lang & type & location\\
	\hline
	i	& front closed unrounded vowel	& English see, Spanish sí, French vite, German mi.e.ten, Italian visto	& xxx	& vowel	& front	\\
	I	& front closed unrounded vowel, but somewhat more centralised and relaxed, in Polish: mid closed unrounded	& English city, German mit	& xxx	& vowel	& front	\\
	1	& close central unrounded vowel	& Russian mys	& xxx	& vowel	& central	\\
	e	& front half closed unrounded vowel	& US English bear, Spanish él, French année, German mehr, Italian rete, Catalan més	& xxx	& vowel	& front	\\
	E	& front half open unrounded vowel	& English bed, French même, German Herr, Männer, Italian ferro, Catalan mes, Spanish perro	& xxx	& vowel	& front	\\
	\{	& front open unrounded vowel	& English cat	& xxx	& vowel	& front	\\
	y	& front closed rounded vowel	& French du, German Tür	& xxx	& vowel	& front	\\
	2	& front half closed rounded vowel	& French deux (hence '2'), German Höhle	& xxx	& vowel	& front	\\
	9	& front half open rounded vowel	& French neuf (hence '9'), German Hölle	& xxx	& vowel	& front	\\
	@	& central neutral unrounded vowel	& English about, winner,German bitte	& xxx	& vowel	& central	\\
	P6	& central neutral unrounded vowel	& German besser	& xxx	& vowel	& central\\
	3	& front half open unrounded vowel, but somewhat more centralised and relaxed	& English bird,nurse	& xxx	& vowel	& central	\\
	a	& central open vowel	& Spanish da, barra, French bateau,lac, German Haar, Italian pazzo	& xxx	& vowel	& front	\\
	\}	& central closed rounded vowel	& Scottish English pool, Swedish sju	& xxx	& vowel	& central	\\
	8	& central neutral rounded vowel	& Swedish kust	& xxx	& vowel	& central	\\
	\&	& front open rounded vowel	& American English that & xxx	& vowel	& front	\\
	M	& back closed unrounded vowel	& Japanese fuji, Korean eu & xxx	& vowel	& back	\\
	7	& back half closed unrounded vowel	& Korean eo & xxx	& vowel	& back	\\
	V	& back half open unrounded vowel	& RP and US English run,enough,strut	& xxx	& vowel	& back	\\
	A	& back open unrounded vowel	& English arm, US English law, standard French âme	& xxx	& vowel	& back	\\
	u	& back closed rounded vowel	& English soon, Spanish tú, French goût, German Hut, Mutter, Italian azzurro,tutto	& xxx	& vowel	& back	\\
	U	& back closed rounded vowel somewhat more centralised and relaxed	& English put, Buddhist	& xxx	& vowel	& back	\\
	o	& back half closed rounded vowel	& US English sore, Scottish English boat, Spanish yo, French beau, German Sohle, Italian dove, Catalan ona	& xxx	& vowel	& back	\\
	O	& back half open rounded vowel	& British English law, caught, Italian cosa, Catalan dona, Spanish ojo, German Wort	& xxx	& vowel	& back	\\
	Q	& back open rounded vowel	& British English not, cough	& xxx	& vowel	& back	\\
	Y	& lax [y]	& German hübsch	& xxx	& vowel	& front	\\
	p	& voiceless bilabial plosive	& English pen	& xxx	& plosive	& bilabial	\\
	b	& voiced bilabial plosive	& English but	& xxx	& plosive	& bilabial	\\
	t	& voiceless alveolar plosive	& English two, Spanish toma	& xxx	& plosive	& alveolar	\\
	d	& voiced alveolar plosive	& English do, Italian cade	& xxx	& plosive	& alveolar	\\
	ts	& voiceless alveolar affricate	& Italian calza, German zeit	& xxx	& affricate	& alveolar	\\
	dz	& voiced alveolar affricate	& Italian zona	& xxx	& affricate	& alveolar	\\
	tS	& voiceless postalveolar affricate	& English chair, , Spanish mucho	& xxx	& affricate	& post-alveolar	\\
	dZ	& voiced postalveolar affricate	& English gin, Italian giorno	& xxx	& affricate	& post-alveolar	\\
	pf	& voiceless labial affricate	& German Pferd	& xxx	& affricate	& bilabial	\\
	c	& voiceless palatal plosive	& Hungarian tyúk 'hen'	& xxx	& plosive	& palatal	\\
	k	& voiceless velar plosive	& English skill	& xxx	& plosive	& velar	\\
	g	& voiced velar plosive	& English go	& xxx	& plosive	& velar	\\
	q	& voiceless uvular plosive	& Arabic qof	& xxx	& plosive	& uvular	\\
	B	& voiced bilabial fricative	& Catalan roba 'clothes'	& xxx	& fricative	& bilabial	\\
	f	& voiceless labiodental fricative	& English fool, Spanish and Italian falso	& xxx	& fricative	& labio-dental	\\
	v	& voiced labiodental fricative	& English voice, German Welt	& xxx	& fricative	& labio-dental	\\
	T	& voiceless dental fricative	& English thing, Castilian Spanish caza	& xxx	& fricative	& dental	\\
	D	& voiced dental fricative	& English this	& xxx	& fricative	& dental\\
	s	& voiceless alveolar fricative	& English see, Spanish sí	& xxx	& fricative	& alveolar	\\
	z	& voiced alveolar fricative	& English zoo, German See	& xxx	& fricative	& alveolar	\\
	S	& voiceless postalveolar fricative	& English she, French chemin	& xxx	& fricative	& post-alveolar	\\
	Z	& voiced postalveolar fricative	& French jour, English pleasure	& xxx	& fricative	& post-alveolar	\\
	C	& voiceless palatal fricative	& German Ich	& xxx	& fricative	& palatal	\\
	x	& voiceless velar fricative	& Scots loch, Castilian Spanish ajo	& xxx	& fricative	& velar	\\
	G	& voiced velar fricative	& Greek $\gamma\alpha\lambda\alpha$	& xxx	& fricative	& velar	\\
	h	& voiceless glottal fricative	& English ham, German Hand	& xxx	& fricative	& glottal	\\
	m	& bilabial nasal	& English man	& xxx	& nasal	& bilabial	\\
	F	& labiodental nasal	& Spanish infierno, Hungarian kámfor	& xxx	& nasal	& labio-dental	\\
	n	& alveolar nasal	& English, Spanish and Italian no	& xxx	& nasal	& alveolar	\\
	J	& palatal nasal	& Spanish año, French oignon	& xxx	& nasal	& palatal	\\
	N	& velar nasal	& English ring, Italian bianco, Tagalog ngayón	& xxx	& nasal	& velar	\\
	l	& alveolar lateral approximant	& English left, Spanish largo	& xxx	& lateral-approximant	& alveolar	\\
	L	& palatal lateral approximant	& Italian aglio, Catalan colla,	& xxx	& lateral-approximant	& palatal	\\
	5	& velarized dental lateral	& English meal Catalan alga	& xxx	& lateral-approximant	& dental-velar	\\
	4	& alveolar tap	& Spanish pero, American English muddy	& use	& tap	& alveolar	\\
	r	& alveolar trill	& Spanish perro	& xxx	& trill	& alveolar	\\
	R	& uvular trill	& German Reich	& xxx	& trill	& uvular	\\
	P	& labiodental approximant	& Dutch Waar	& xxx	& approximant	& labio-dental	\\
	w	& labial-velar approximant	& English we, French oui	& xxx	& approximant	& labio-velar	\\
	H	& labial-palatal approximant	& French huit	& xxx	& approximant	& labio-palatal	\\
	j	& palatal approximant	& English yes, French yeux	& xxx	& approximant	& palatal	\\
	?	& glottal stop	& German Verein, Danish stød	& xxx	& plosive	& glottal	\\
	W	& voiceless labial-velar fricative	& 	& xxx	& fricative	& labio-velar	\\
	D:	& lengthened voiced dental fricative	& 	& fin	& fricative	& dental	\\
	T:	& lengthened voiceless dental fricative	& 	& fin	& fricative	& dental	\\
	i:	& lengthened front closed unrounded vowel	& mieten	& deu	& vowel	& front	\\
	ii	& lengthened front closed unrounded vowel in quantity II	& riisu	& ekk	& vowel	& front	\\
	ii:	& lengthened front closed unrounded vowel in quantity III	& 	& ekk	& vowel	& front	\\
	e:	& lengthened front half closed unrounded vowel	& mehr	& deu	& vowel	& front	\\
	ee	& lengthened front half closed unrounded vowel in quantity II	& keere	& ekk	& vowel	& front	\\
	ee:	& lengthened front half closed unrounded vowel in quantity III	& 	& ekk	& vowel	& front	\\
	E:	& lengthened front half open unrounded vowel	& Mär	& deu	& vowel	& front	\\
	y:	& lengthened front closed rounded vowel	& Tür	& deu	& vowel	& front	\\
	Y:	& lengthened lax [y]	& (Swiss German)	& deu	& vowel	& front	\\
	2:	& lengthened front half closed rounded vowel	& Höhle	& deu	& vowel	& front	\\
	a:	& lengthened central open vowel	& Haar	& deu	& vowel	& central	\\
	u:	& lengthened back closed rounded vowel	& Hut	& deu	& vowel	& back	\\
	o:	& lengthened back half closed rounded vowel	& Sohle	& deu	& vowel	& back	\\
	3:	& lengthened front half open unrounded vowel	& furs	& aus	& vowel	& front	\\
	A:	& lengthened back open unrounded vowel	& stars	& aus	& vowel	& back	\\
	O:	& lengthened back half open rounded vowel	& cause	& aus	& vowel	& back	\\
	P6:	& lengthened central neutral unrounded vowel	& 	& aus	& vowel	& central	\\
	\}:	& lengthened central closed rounded vowel	& pool	& aus	& vowel	& central	\\
	Q:	& lengthened open back rounded	& (Swiss German)	& aus	& vowel	& back	\\
	9:	& lengthened front half open rounded vowel	& 	& nld	& vowel	& front	\\
	\{\{	& lengthened front open unrounded vowel in quantity II	& kääru	& ekk	& vowel	& front	\\
	\{:	& lengthened front open unrounded vowel in quantity II	& kääru	& ekk	& vowel	& front	\\
	\{\{:	& lengthened front open unrounded vowel in quantity III	& 	& ekk	& vowel	& front	\\
	yy	& lengthened front closed rounded vowel (quantitiy II)	& müüri	& ekk	& vowel	& front	\\
	yy:	& lengthened front closed rounded vowel (quantitiy III)	& 	& ekk	& vowel	& front	\\
	22	& lengthened front half closed rounded vowel	& nööri	& ekk	& vowel	& front	\\
	22:	& lengthened front half closed rounded vowel in quantity III	& 	& ekk	& vowel	& front	\\
	uu	& lengthened back closed rounded vowel	& kuuri	& ekk	& vowel	& back	\\
	uu:	& lengthened back closed rounded vowel in quantity III	& 	& ekk	& vowel	& back	\\
	oo	& lengthened back half closed rounded vowel	& poori	& ekk	& vowel	& back	\\
	oo:	& lengthened back half closed rounded vowel in quantity III	& 	& ekk	& vowel	& back	\\
	77	& back half closed unrounded vowel in quantity II	& sõõre	& ekk	& vowel	& back	\\
	7:	& back half closed unrounded vowel in quantity II	& sõõre	& ekk	& vowel	& back	\\
	77:	& back half closed unrounded vowel in quantity III	& 	& ekk	& vowel	& back	\\
	AA	& lengthened back open unrounded vowel in quantity II	& vaaru	& ekk	& vowel	& back	\\
	AA:	& lengthened back open unrounded vowel in quantity III	& 	& ekk	& vowel	& back	\\
	aU	& diphthong	& Haus	& deu	& diphthong	& front\textgreater back	\\
	aI	& diphthong	& Bein	& deu	& diphthong	& front	\\
	ai	& diphthong	& 	& ita	& diphthong	& front	\\
	a:i	& diphthong	& 	& nld	& diphthong	& front	\\
	Ae	& diphthong	& 	& aus	& diphthong	& back\textgreater front	\\
	Au	& diphthong	& 	& nld	& diphthong	& back	\\
	OY	& diphthong	& heulen	& deu	& diphthong	& back\textgreater front	\\
	eI	& diphthong	& raise	& aus	& diphthong	& front	\\
	e@	& diphthong	& stairs	& aus	& diphthong	& front\textgreater central	\\
	ei	& diphthong	& 	& ita	& diphthong	& front	\\
	e:i	& diphthong	& 	& nld	& diphthong	& front	\\
	eU	& diphthong	& 	& por	& diphthong	& front\textgreater back	\\
	EI	& diphthong	& raise	& eng	& diphthong	& front	\\
	Ei	& diphthong	& 	& ita	& diphthong	& front	\\
	\{I	& diphthong	& 	& aus	& diphthong	& front	\\
	\{O	& diphthong	& 	& aus	& diphthong	& front\textgreater back	\\
	I@	& diphthong	& dears	& aus	& diphthong	& front\textgreater central	\\
	Ii:	& diphthong	& accede	& aus	& diphthong	& front	\\
	I:	& lengthened front closed unrounded vowel	& (Swiss German)	& deu	& vowel	& front	\\
	i:@	& diphthong	& memorial	& nze	& diphthong	& central\textgreater front	\\
	io	& diphthong	& 	& ita	& diphthong	& front\textgreater back	\\
	iu	& diphthong	& 	& nld	& diphthong	& front\textgreater back	\\
	ja	& diphthong	& 	& ita	& diphthong	& front	\\
	jo	& diphthong	& 	& ita	& diphthong	& front\textgreater back	\\
	ju	& diphthong	& 	& ita	& diphthong	& front\textgreater back	\\
	oI	& diphthong	& 	& aus	& diphthong	& back\textgreater front	\\
	oi	& diphthong	& 	& ita	& diphthong	& back\textgreater front	\\
	o:i	& diphthong	& 	& nld	& diphthong	& back\textgreater front	\\
	oU	& diphthong	& 	& por	& diphthong	& back	\\
	oE	& diphthong	& 	& ita	& diphthong	& back\textgreater front	\\
	OI	& diphthong	& noise	& aus	& diphthong	& back\textgreater front	\\
	Oi	& diphthong	& 	& ita	& diphthong	& back\textgreater front	\\
	ue	& diphthong	& 	& ita	& diphthong	& back\textgreater front	\\
	ui	& diphthong	& 	& nld	& diphthong	& back\textgreater front	\\
	U@	& diphthong	& cures	& aus	& diphthong	& back\textgreater central	\\
	U:	& lenthened back closed rounded vowel somewhat more centralised and relaxed	& (Swiss German)	& deu	& vowel	& back	\\
	wa	& diphthong	& 	& ita	& diphthong	& front	\\
	we	& diphthong	& 	& ita	& diphthong	& front	\\
	wi	& diphthong	& 	& ita	& diphthong	& front	\\
	wO	& diphthong	& 	& ita	& diphthong	& back	\\
	yu	& diphthong	& 	& nld	& diphthong	& front\textgreater back	\\
	@U	& diphthong	& nose	& aus	& diphthong	& central\textgreater back	\\
	@\}	& diphthong	& 	& aus	& diphthong	& central	\\
	@@	& geminate of schwa in quantity II	& 	& ekk	& vowel	& central	\\
	@:	& geminate of schwa in quantity II	& 	& ekk	& vowel	& central	\\
	@@:	& geminate of schwa in quantity III	& 	& ekk	& vowel	& central	\\
	9y	& diphthong	& 	& nld	& diphthong	& back\textgreater front	\\
	QI	& diphthong	& abide	& aus	& diphthong	& central\textgreater front	\\
	U\}	& diphthong	& abuse	& aus	& diphthong	& central\textgreater back	\\
	VU	& diphtong	& acetone	& aus	& diphthong	& central\textgreater back	\\
	Vi	& diphthong	& abased	& aus	& diphthong	& central\textgreater front	\\
	\{o	& diphthong	& accounts	& aus	& diphthong	& front\textgreater back	\\
	2i	& diphthong	& 	& ekk	& diphthong	& back\textgreater front	\\
	2i:	& diphthong	& 	& ekk	& diphthong	& back\textgreater front	\\
	7o:	& diphthong	& 	& ekk	& diphthong	& back	\\
	7u:	& diphthong	& 	& ekk	& diphthong	& back	\\
	7u	& diphthong	& 	& ekk	& diphthong	& back	\\
	Ai	& diphthong	& 	& ekk	& diphthong	& back\textgreater front	\\
	Ai:	& diphthong	& 	& ekk	& diphthong	& back\textgreater front	\\
	Ao:	& diphthong	& 	& ekk	& diphthong	& back	\\
	Au:	& diphthong	& 	& ekk	& diphthong	& back	\\
	Ae:	& diphthong	& 	& ekk	& diphthong	& back\textgreater front	\\
	ei:	& diphthong	& 	& ekk	& diphthong	& front	\\
	e:u	& diphthong	& 	& nld	& diphthong	& front	\\
	i~	& nasalized front closed unrounded vowel	& 	& xxx	& vowel	& front	\\
	e~	& nasalized front half closed unrounded vowel	& vin	& fra	& vowel	& front	\\
	a~	& nasalized central open vowel	& vent	& fra	& vowel	& front	\\
	o~	& nasalized back half closed rounded vowel	& bon	& fra	& vowel	& back	\\
	9~	& nasalized front half open rounded vowel	& brun,neuv	& fra	& vowel	& front	\\
	E~	& nasalized lengthened front half open unrounded vowel	& 	& deu	& vowel	& front	\\
	O~	& nasalized back half open rounded vowel	& 	& deu	& vowel	& back	\\
	u~	& nasalized back closed rounded vowel	& 	& xxx	& vowel	& back	\\
	a:~	& nasalized lengthened central open vowel	& 	& deu	& vowel	& central	\\
	E:~	& nasalized lengthened front half open unrounded vowel	& 	& deu	& vowel	& front	\\
	o:~	& nasalized lengthened back half closed rounded vowel	& 	& deu	& vowel	& back	\\
	O:~	& nasalized lengthened back half open rounded vowel	& 	& nld	& vowel	& back	\\
	ts\_j	& palatalized voiceless alveolar affricate	& c'ma	& pol	& affricate	& post-alveolar	\\
	dz\_j	& palatalized voiced alveolar affricate	& dz'wig	& pol	& affricate	& alveolar	\\
	s\_j	& palatalized voiceless alveolar fricative	& syk	& pol	& fricative	& alveolar	\\
	s\_js	& palatalized voiceless alveolar fricative in quantity II	& kassi	& ekk	& fricative	& alveolar	\\
	s\_j:s	& palatalized voiceless alveolar fricative in quantity III	& 	& ekk	& fricative	& alveolar	\\
	z\_j	& palatalized voiced alveolar fricative	& zbir	& pol	& fricative	& alveolar	\\
	n\_j	& palatalized alveolar nasal	& kon'	& pol	& nasal	& alveolar	\\
	n\_jn	& palatalized alveolar nasal in quantity II	& panni	& ekk	& nasal	& alveolar	\\
	n\_j:n	& palatalized alveolar nasal in quantity III	& 	& ekk	& nasal	& alveolar	\\
	l\_j	& palatalized alveolar lateral approximant	& pali	& ekk	& lateral-approximant	& alveolar	\\
	l\_jl	& palatalized alveolar lateral approximant in quantity II	& palli	& ekk	& lateral-approximant	& alveolar	\\
	l\_j:l	& palatalized alveolar lateral approximant in quantity III	& 	& ekk	& lateral-approximant	& alveolar	\\
	t\_j	& palatalized voiceless alveolar plosive	& padi	& ekk	& plosive	& alveolar	\\
	t\_jt	& palatalized voiceless alveolar plosive in quantity II	& pati	& ekk	& plosive	& alveolar	\\
	t\_j:t	& palatalized voiceless alveolar plosive in quantity III	& 	& ekk	& plosive	& alveolar	\\
	d\_j	& palatalized voiced alveolar plosive	& gyár	& hun	& plosive	& alveolar	\\
	dd\_j	& geminate of d'	& egy	& hun	& plosive	& alveolar	\\
	g\_j	& palatalized voiced velar plosive	& Gienek	& pol	& plosive	& velar	\\
	x\_j	& palatalized voiceless velar fricative	& hiacynt	& pol	& fricative	& velar	\\
	k\_j	& palatalized voiceless velar plosive	& kierowca	& pol	& plosive	& velar	\\
	p\_j	& palatalized voiceless bilabial plosive	& piasek	& pol	& plosive	& bilabial	\\
	xx\_j	& geminate of x'	& 	& hun	& fricative	& velar	\\
	p\_h	& aspirated voiceless bilabial plosive	& 	& spa	& plosive	& bilabial	\\
	t\_h	& aspirated voiceless alveolar plosive	& 	& spa	& plosive	& alveolar	\\
	k\_h	& aspirated voiceless velar plosive	& 	& spa	& plosive	& velar	\\
	tt	& geminate of t	& fatto	& ita	& fricative	& bilabial	\\
	t:	& lengthened t	& että	& fin	& plosive	& alveolar	\\
	t:t	& geminate of t in quantity III	& 	& ekk	& plosive	& alveolar	\\
	pp	& geminate of p	& 	& ita	& plosive	& bilabial	\\
	p:	& lengthened p	& 	& fin	& plosive	& bilabial	\\
	p:p	& geminate of p in quantity III	& 	& ekk	& plosive	& bilabial	\\
	kk	& geminate of k	& 	& ita	& plosive	& velar	\\
	k:	& lengthened k	& takkinsa	& fin	& plosive	& velar	\\
	k:k	& geminate of k in quantity III	& 	& ekk	& plosive	& velar	\\
	dd	& geminate of d	& 	& ita	& plosive	& alveolar	\\
	gg	& geminate of g	& 	& ita	& plosive	& velar	\\
	g:	& lengthened g	& 	& fin	& plosive	& velar	\\
	bb	& geminate of b	& 	& ita	& plosive	& bilabial	\\
	b:	& lengthened b	& 	& fin	& plosive	& bilabial	\\
	ttS	& geminate of tS	& 	& ita	& affricate	& post-alveolar	\\
	tts	& geminate of ts	& 	& ita	& affricate	& alveolar	\\
	ddZ	& geminate of dZ	& 	& ita	& affricate	& post-alveolar	\\
	ddz	& geminate of dz	& zona	& ita	& affricate	& alveolar	\\
	vv	& geminate of v	& 	& ita	& fricative	& labio-dental	\\
	v:	& lengthened v	& 	& fin	& fricative	& labio-dental	\\
	v:v	& geminate of v in quantity III	& 	& ekk	& fricative	& labio-dental	\\
	ss	& geminate of s	& 	& ita	& fricative	& alveolar	\\
	s:	& lengthened s	& 	& fin	& fricative	& alveolar	\\
	s:s	& geminate of s in quantity III	& 	& ekk	& fricative	& alveolar	\\
	zz	& geminate of z	& 	& hun	& fricative	& alveolar	\\
	SS	& geminate of S	& 	& ita	& fricative	& post-alveolar	\\
	S:	& lengthened S	& 	& fin	& fricative	& post-alveolar	\\
	S:S	& geminate of S in quantity III	& 	& ekk	& fricative	& post-alveolar	\\
	ZZ	& geminate of Z	& 	& hun	& fricative	& post-alveolar	\\
	xx	& geminate of x	& 	& hun	& fricative	& velar	\\
	rr	& geminate of r	& 	& ita	& trill	& alveolar	\\
	r:	& lengthened r	& 	& fin	& trill	& alveolar	\\
	r:r	& geminate of r in quantity III	& 	& ekk	& trill	& alveolar	\\
	RR	& geminate of R in quantity II	& 	& ekk	& trill	& uvular	\\
	nn	& geminate of n	& 	& ita	& nasal	& alveolar	\\
	n:	& lengthened n	& 	& fin	& nasal	& alveolar	\\
	n:n	& geminate of n in quantity III	& 	& ekk	& nasal	& alveolar	\\
	NN	& geminate of N	& (Swiss German)	& deu	& nasal	& velar	\\
	N:	& lengthened N	& 	& fin	& nasal	& velar	\\
	N\_j	& palatalized velar nasal	& 	& ekk	& nasal	& velar	\\
	ww	& geminate of w	& (Swiss German)	& deu	& approximant	& labio-velar	\\
	mm	& geminate of m	& 	& ita	& nasal	& bilabial	\\
	m:	& lengthened m	& hommasta	& fin	& nasal	& bilabial	\\
	m:m	& geminate of m in quantity III	& 	& ekk	& nasal	& bilabial	\\
	FF	& geminate of F	& 	& hun	& nasal	& labio-dental	\\
	LL	& geminate of L	& 	& ita	& lateral-approximant	& palatal	\\
	ll	& geminate of l	& 	& ita	& lateral-approximant	& alveolar	\\
	l:	& lengthened l	& jolla	& fin	& lateral-approximant	& alveolar	\\
	l:l	& geminate of l in quantity III	& 	& ekk	& approximant	& alveolar	\\
	JJ	& geminate of J	& 	& ita	& nasal	& palatal	\\
	jj	& geminate of j	& 	& ekk	& approximant	& palatal	\\
	j:	& lengthened j	& 	& fin	& approximant	& palatal	\\
	j:j	& geminate of j in quantity III	& 	& ekk	& approximant	& palatal	\\
	ff	& geminate of f	& 	& ita	& fricative	& bilabial	\\
	f:	& lengthened f	& 	& fin	& fricative	& bilabial	\\
	f:f	& geminate of f in quantity III	& 	& ekk	& fricative	& bilabial	\\
	hh	& geminate of h	& 	& ekk	& fricative	& glottal	\\
	h:	& lengthened h	& 	& fin	& fricative	& glottal	\\
	h:h	& geminate of h in quantity III	& 	& ekk	& fricative	& glottal	\\
	ttS\_cl	& closure of ttS	& 	& ita	& affricate	& post-alveolar	\\
	ttS\_rl	& release of ttS	& 	& ita	& affricate	& post-alveolar	\\
	tts\_cl	& closure of tts	& 	& ita	& affricate	& alveolar	\\
	tts\_rl	& release of tts	& 	& ita	& affricate	& alveolar	\\
	ddZ\_cl	& closure of ddZ	& 	& ita	& affricate	& post-alveolar	\\
	ddZ\_rl	& release of ddZ	& 	& ita	& affricate	& post-alveolar	\\
	ddz\_cl	& closure of ddz	& zona	& ita	& affricate	& alveolar	\\
	ddz\_rl	& release of ddz	& zona	& ita	& affricate	& alveolar	\\
	tS\_cl	& closure of tS	& 	& ita	& affricate	& post-alveolar	\\
	tS\_rl	& release of tS	& 	& ita	& affricate	& post-alveolar	\\
	ts\_cl	& closure of ts	& 	& ita	& affricate	& alveolar	\\
	ts\_rl	& release of ts	& 	& ita	& affricate	& alveolar	\\
	dZ\_cl	& closure of dZ	& 	& ita	& affricate	& post-alveolar	\\
	dZ\_rl	& release of dZ	& 	& ita	& affricate	& post-alveolar	\\
	dz\_cl	& closure of dz	& 	& ita	& affricate	& alveolar	\\
	dz\_rl	& release of dz	& 	& ita	& affricate	& alveolar	\\
	tt\_rl	& release of tt	& 	& ita	& plosive	& alveolar	\\
	tt\_cl	& closure of tt	& 	& ita	& plosive	& alveolar	\\
	pp\_cl	& closure of pp	& 	& ita	& plosive	& bilabial	\\
	pp\_rl	& release of pp	& 	& ita	& plosive	& bilabial	\\
	kk\_cl	& closure of kk	& 	& ita	& plosive	& velar	\\
	kk\_rl	& release of kk	& 	& ita	& plosive	& velar	\\
	dd\_cl	& release of dd	& 	& ita	& plosive	& alveolar	\\
	dd\_rl	& release of dd	& 	& ita	& plosive	& alveolar	\\
	gg\_cl	& release of gg	& 	& ita	& plosive	& velar	\\
	gg\_rl	& release of gg	& 	& ita	& plosive	& velar	\\
	bb\_cl	& release of bb	& 	& ita	& plosive	& bilabial	\\
	bb\_rl	& release of bb	& 	& ita	& plosive	& bilabial	\\
	t\_cl	& closure of t	& 	& ita	& plosive	& alveolar	\\
	t\_rl	& release of t	& 	& ita	& plosive	& alveolar	\\
	p\_cl	& closure of p	& 	& ita	& plosive	& bilabial	\\
	p\_rl	& release of p	& 	& ita	& plosive	& bilabial	\\
	k\_cl	& closure of k	& 	& ita	& plosive	& velar	\\
	k\_rl	& release of k	& 	& ita	& plosive	& velar	\\
	g\_cl	& closure of g	& 	& ita	& plosive	& velar	\\
	g\_rl	& release of g	& 	& ita	& plosive	& velar	\\
	d\_cl	& closure of d	& 	& ita	& plosive	& alveolar	\\
	d\_rl	& release of d	& 	& ita	& plosive	& alveolar	\\
	b\_cl	& closure of b	& 	& ita	& plosive	& bilabial	\\
	b\_rl	& release of b	& 	& ita	& plosive	& bilabial	\\
	\textless	& recording initial silence	& 	& xxx	& silence	& silence	\\
	\textgreater	& recording trailing silence	& 	& xxx	& silence	& silence	\\
	\#	& inter-word silence	& 	& xxx	& silence	& silence	\\
	\textless nib\textgreater	& noise, non-human	& 	& xxx	& noise	& noise	\\
	\textless p:\textgreater	& silence interval	& 	& xxx	& silence	& silence	\\
	\textless usb\textgreater	& human noise, garbage	& 	& xxx	& noise	& noise	\\
	p\_\textgreater	& voiceless bilabial ejective	& Georgian p'erangi	& xxx	& ejective	& bilabial\\	
	t\_\textgreater	& voiceless alveolar ejective	& Georgian zedat'ani	& xxx	& ejective	& alveolar\\
	ts\_\textgreater	& voiceless alveolar affricate ejective	& Georgian ts'indebi	& xxx	& ejective	& alveolar	\\
	ts`	& retroflex voiceless affricate	& cyk	& pol	& affricate	& retroflex	\\
	dz`	& retroflex voiced affricate	& dzwon	& pol	& affricate	& retroflex	\\
	3`	& retroflex front half open unrounded vowel	& furs	& use	& vowel	& front	\\
	tS\_\textgreater	& voiceless postalveolar affricate ejective	& Georgian k'uch'is	& xxx	& ejective	& post-alveolar	\\
	c\_\textgreater	& voiceless palatal ejective plosive	& Georgian ch' & xxx	& ejective	& palatal	\\
	J-	& voiced palatal plosive	& Hungarian egy 'one'	& xxx	& plosive	& palatal	\\
	k\_\textgreater	& voiceless velar ejective	& Georgian uk'an	& xxx	& ejective	& velar	\\
	q\_\textgreater	& voiceless uvular ejective	& Georgian saavadmq'opo	& xxx	& ejective	& uvular\\
	p-	& voiceless bilabial fricative	& Japanese fu	& xxx	& fricative	& bilabial	\\
	s`	& retroflex voiceless fricative	& szyk	& pol	& fricative	& retroflex	\\
	z`	& retroflex voiced fricative	& żyto	& pol	& fricative	& retroflex	\\
	j-	& voiced palatal fricative	& Spanish ayuda	& xxx	& fricative	& palatal	\\
	M-	& velar approximant	& Spanish algo	& xxx	& approximant	& velar	\\
	X-	& voiceless pharyngeal fricative	& Arabic h.â	& xxx	& fricative	& pharyngeal	\\
	?-	& voiced pharyngeal fricative	& Arabic 'ayn	& xxx	& fricative	& pharyngeal	\\
	h-	& voiced glottal fricative	& English	& use	& fricative	& glottal	\\
	r-	& alveolar approximant	& English run	& xxx	& approximant	& alveolar	\\
	N=	& syllabic velar nasal	& English walking	& use	& nasal	& velar	\\
	n=	& syllabic alveolar nasal	& English Boston	& use	& nasal	& alveolar	\\
	m=	& syllabic bilabial nasal	& English bottom	& use	& nasal	& bilabial	\\
	l=	& syllabic lateral	& English bridal	& use	& lateral-approximant	& alveolar	\\
\end{longtable}
}

\newpage
\section{Version history}
\begin{longtable}{|p{0.22\linewidth}p{0.8\linewidth}|}
	\hline
	1.8 (2015-11-19) & \tabitem Added support for \texttt{Sound}.\\
		& \tabitem Changed email address of author.\\
		& \tabitem Elaborated the installation instructions.\\
		& \tabitem Splitted up the model and phonetizer selection.\\
		& \tabitem Cleaned up the phone specifications.\\
		& \tabitem Added an universal phonetizer.\\
		& \tabitem Big book update with structural changes.\\
	\hline
	1.7a (2015-10-13) & \tabitem Fixed critical tier creation bug.\\
	& \tabitem Added version to settings window.\\
	\hline
	1.7 (2015-10-08) & \tabitem More robuust tier creation.\\
		& \tabitem Some typos fixed in the manual.\\
		& \tabitem Updated the bibtex snippet.\\
		&	\tabitem Better tier alignment when the tier is empty.\\
		&	\tabitem Corrected some spelling errors.\\
		&	\tabitem Tidied the makefile for the book.\\
	\hline
	1.6a (2015-09-29) & \tabitem Fixed the dictionary generator for English.\\
		& \tabitem Correctly added websites to the authors of the models.\\
	\hline
	1.60 (2015-09-07) & \tabitem Fixed menu name from \texttt{force} to
\texttt{forced}.\\
		& \tabitem Fixed silence problem in experimental models.\\
		& \tabitem Added version to setup dialog.\\
		& \tabitem Added language for using the general sampa model without
			phonetizer.\\
		& \tabitem Created a better error when loading rulesets or dictionaries in
			non supported encodings or non existent files.\\
		& \tabitem Changed the licence to MIT.\\
	\hline
	1.50 (2015-07-07) & \tabitem Spanish models merged in original master.\\
		& \tabitem Procedures in different file, thus cleaner code.\\
		& \tabitem Reinitialized repo.\\
		& \tabitem Faster book compilation.\\
	\hline
	1.40 (2015-07-01) & \tabitem Added branch for own spanish models.\\
		& \tabitem Updated book about ruleset.\\
		& \tabitem Changed name in menu from \texttt{Setup\ldots} to 
			\texttt{Set up\dots}.\\
	\hline
	1.39 (2015-04-01) & \tabitem Fixed tier alignment.\\
	\hline
	1.3 (2015-03-25) & \tabitem Added option for custom python path.\\
	 & \tabitem Greatly simplified rulesets.\\
	\hline
	1.2 (2015-02-18) & \tabitem Simplified installation scripts.\\
		& \tabitem Fixed a unicode bug in generating dictionaries.\\
	\hline
	1.1 (2015-01-30) & \tabitem Implemented better error messaging.\\
		& \tabitem Simplified the python code into one file.\\
		& \tabitem Fixed a bug that leaded to a messed up view.\\
		& \tabitem Fixed small phonetization errors.\\
	\hline
	1.0 (2015-01-09) & \tabitem Converted the readme to a pdf.\\
		& \tabitem Speeded up the process by disabling pitch, intensity,
spectrum, pulses and formants while aligning, the settings do get restored
afterwards.\\
		& \tabitem Fixed a bug that leaded to a messed up view.\\
	\hline
	0.9a (2014-12-02) & \tabitem Small bugfix in dictionary generation fixed.\\
	\hline
	0.9 & \tabitem Cleaned up some parts of the readme.\\
		& \tabitem Added language specific information.\\
		& \tabitem Added english as language. Although there is no phonetizing
implemented.\\
		& \tabitem README.html better with light background for code blocks.\\
		& \tabitem Updated citing method with bibtex.\\
	\hline
	0.8 (2014-10-31) & \tabitem Removed all the binary folders.\\
		& \tabitem Made the binary finding interactive.\\
		& \tabitem Made all the file chooser dialogs interactive.\\
	\hline
	0.7 (2014-10-29) & \tabitem Added windows support.\\
		&	\tabitem Cleaned up documentation.\\
		& \tabitem Removed binaries due htk licence.\\
	\hline
	0.6 (2014-10-22) & \tabitem Refactored and cleaned up the source.\\
	\hline
	0.5a (2014-09-08) & \tabitem Added comments to source code(praat).\\
		& \tabitem Cleaned up source.\\
	\hline
	0.5 (2014-09-04) & \tabitem Fixed acronyms in spanish.\\
		& \tabitem Fixed cleaning with extended boundaries.\\
		& \tabitem Added rudimentary ruleset implementation.\\
	\hline
	0.4 (2014-08-29) & \tabitem Added option for enlarging the boundaries
automatically.\\
	\hline
	0.21 (2014-08-13) & \tabitem Settings split in non interactive and
interactive so that the interactive one reflects the current settings.\\
	\hline
	0.2 (2014-08-11) & \tabitem Better mac compatibility.\\
	\hline
	0.1a (2014-06-30) & \tabitem Tier alignment fixed.\\
		& \tabitem Readme for dutch.\\
	\hline
	0.08 (2014-04-29) & \tabitem Cleaned up some stuff.\\
		& \tabitem Added dutch.\\
		& \tabitem Readme for spanish and sampa.\\
	\hline
	0.07 (2014-04-28) & \tabitem Non interactive alignment implemented.\\
		& \tabitem Table of contents in readme.\\
	\hline
	0.06 (2014-04-25) & \tabitem Conversion to editor scripts.\\
	\hline
	0.05 (2014-04-03) & \tabitem Better readme.\\
		& \tabitem Functional program for linux.\\
	\hline
	0.04 (2014-04-03) & \tabitem Pronunciation variants implemented.\\
	\hline
	0.03 (2014-03-31) & \tabitem Aligner works in python.\\
	\hline
	0.02 (2014-03-27) & \tabitem Python script around aligner started.\\
		& \tabitem Phonetizer skeleton done.\\
	\hline
	\caption{Version history}
\end{longtable}

\bibliographystyle{ieeetr}
\bibliography{book}

\end{document}
